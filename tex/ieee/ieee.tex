\documentclass[lettersize,journal]{IEEEtran}
\usepackage{amsmath,amsfonts}
\usepackage{algorithmic}
\usepackage{algorithm}
\usepackage{array}
\usepackage[caption=false,font=normalsize,labelfont=sf,textfont=sf]{subfig}
\usepackage{textcomp}
\usepackage{stfloats}
\usepackage{url}
\usepackage{verbatim}
\usepackage{graphicx}
\usepackage{cite}

% added packages from authors
\usepackage{xcolor}
\usepackage{booktabs} % for \midrule in tables
\usepackage{hhline}
\usepackage{bbm}
\usepackage{bm}
\usepackage{makecell}
\usepackage[mode=buildnew]{standalone}
\usepackage{tikz}
\usetikzlibrary{calc,arrows,patterns,intersections}
\usepackage{pgfplots}
\usepackage{xcolor}
\usepgfplotslibrary{fillbetween}
\usepackage[breaklinks]{hyperref}
\newcommand{\red}[1]{\textcolor{red}{#1}}
\newcommand{\blue}[1]{\textcolor{black}{#1}}

\hyphenation{op-tical net-works semi-conduc-tor IEEE-Xplore}



%\title{Load Shifting Versus Manual Frequency Reserve: \\ Which One is More Appealing to Flexible Loads?}

%\author{Peter A.V. Gade\textsuperscript{*}\textsuperscript{\textdagger}, Trygve Skjøtskift\textsuperscript{\textdagger}, Charalampos Ziras\textsuperscript{*}, Henrik W. Bindner\textsuperscript{*}, Jalal Kazempour\textsuperscript{*} \\
%    \textsuperscript{*}Department of Wind and Energy Systems, Technical University of Denmark, Kgs. Lyngby, Denmark \\
%    \textsuperscript{\textdagger}IBM Client Innovation Center, Copenhagen, Denmark
    % <-this % stops a space
%    \thanks{Corresponding author. Tel.: +45 24263865. \\ Email addresses: pega@dtu.dk (P.A.V. Gade), Trygve.Skjotskift@ibm.com (T. Skjøtskift), hwbi@dtu.dk (H.W. Bindner), jalal@dtu.dk (J. Kazempour).}% <-this % stops a space
%    \thanks{Manuscript received April 19, 2021; revised August 16, 2021.}}

\IEEEaftertitletext{\vspace{-0.5cm}}
\begin{document}
\bstctlcite{IEEEexample:BSTcontrol}
\title{Load Shifting Versus Manual Frequency Reserve: \\ Which One is More Appealing to Flexible Loads?}

\author{
    \IEEEauthorblockN{%
        Peter A. V. Gade,
        Trygve Skjøtskift,
        Charalampos Ziras,
        Henrik W. Bindner, and
        Jalal Kazempour%
    }
    \vspace{-0.38cm}
    %\IEEEauthorblockA{\IEEEauthorrefmark{1} These authors contributed equally to this work.}
    \thanks{P. A. V. Gade, C. Ziras, H. W. Bindner and J. Kazempour are with the Technical University of Denmark , 2800 Kgs. Lyngby, Denmark (e-mail: \{pega, chazi, hwbi, jalal\}@dtu.dk).}
    \thanks{P. A. V. Gade and T. Skjøtskift are with IBM Client Innovation Center, Copenhagen, Denmark (e-mail: Trygve.Skjotskift@ibm.com)}
    \thanks{This work  was supported in part by Innovation Fund Denmark, Grant Agreement No. 0153-00205B.}
}


% \footnote{Corresponding author. Tel.: +45 24263865. \\ Email addresses: pega@dtu.dk (P.A.V. Gade), Trygve.Skjotskift@ibm.com (T. Skjøtskift), hwbi@dtu.dk (H.W. Bindner), jalal@dtu.dk (J. Kazempour).}

% The paper headers
%\markboth{Journal of \LaTeX\ Class Files,~Vol.~14, No.~8, August~2021}%
%{Shell \MakeLowercase{\textit{et al.}}: A Sample Article Using IEEEtran.cls for IEEE Journals}

%\IEEEpubid{0000--0000/00\$00.00~\copyright~2021 IEEE}
% Remember, if you use this you must call \IEEEpubidadjcol in the second
% column for its text to clear the IEEEpubid mark.

\maketitle

% \tableofcontents

\begin{abstract}
    % \section*{Abstract}
This paper investigates how a  \blue{thermostatically controlled} load can deliver flexibility \blue{either in form of manual frequency reserves or load shifting, and which one is financially more appealing for such a load.} %We discuss advantages and disadvantages of the two and their appeal from a monetary point of view. 
\blue{A supermarket freezer is consideblue as a representative flexible load, and a grey-box model  describing its temperature dynamics is developed} using real data from a supermarket in Denmark. \blue{Taking into account price and activation uncertainties,} a two-stage stochastic mixed-integer linear program is formulated to maximize the flexibility value from the freezer, where two solution strategies for manual frequency reserves are presented: one with a simple policy and \blue{the other one} with a dynamically updated policy. The \blue{proposed} model accurately captures the timeline of decisions for bidding in manual frequency reserves.
%by using McCormick relaxation. 
\blue{Examined on an ex-post out-of-sample simulation,} load shifting shows to be more profitable than \blue{the provision of} manual frequency reserves, but is also more consequential for the temperature in the freezer than flexibility provision for manual frequency reserves.

\end{abstract}

\begin{IEEEkeywords}
    Demand-side flexibility, thermostatically controlled loads, manual frequency reserves, load shifting.
\end{IEEEkeywords}

\vspace{-0.2cm}
\section{Introduction}

Power consumers are looking to reduce costs by any means due to sky-rocketing energy and gas prices. In Denmark, supermarkets are especially exposed due to their high energy consumption, and they are actively looking into initiatives to reduce their electricity bill. One of these initiatives is the application of demand-side flexibility, whereby a supermarket can shift consumption in time when spot (day-ahead) prices are comparatively lower or when the power system needs up-regulation services. There are thousands of supermarkets in Denmark willing to provide flexibility, and their freezers are especially suited for this as they constitute a big share of their energy consumption.

IBM is currently developing a software platform, called \textit{Flex Platform}, aiming to harness flexibility from industrial and commercial consumers for bidding in ancillary service markets via a Balance Responsible Party (BRP). One large supermarket chain in Denmark has already signed up for the \textit{Flex Platform}, and another supermarket chain is currently exploring the possibility of joining.
While these supermarket chains are keen on the provision of flexibility, the estimation of their monetary benefit still remains an open question. We aim to address it by considering two potential ways for supermarkets to exploit their consumption flexibility: (\textit{i}) load shifting, and (\textit{ii}) the provision of manual Frequency Restoration Reserve (mFRR)
%\footnote{The term will be used for the remainder of the paper and is equivalent to tertiary reserves.})
services. We  develop  a stochastic optimization tool capturing the temperature dynamics and therefore consumption flexibility of supermarkets, and compute the monetary benefit of flexibility implemented via load shifting or mFRR service provision. We use two scenario generation  strategies: one using a five-day lookback on spot and balancing market prices in Denmark, and the other one using those price data in 2021. We then systematically compare these two strategies ex-post via an out-of-sample simulation based on price data in 2022.

%In this paper, we describe how to utilize a supermarket freezer's flexibility with respect to manual frequency restoration reserves (mFRR\footnote{The term will be used for the remainder of the paper and is equivalent to tertiary reserves.}) and load shifting, and thereby the incentives for doing so.

%Many studies have used grey-box models to capture the most fundamental physics while using data to calibrate the model. A grey-box model can illustrate the concept of flexibility well, and we show how it can be used for simulations. Such a model can also readily be used in a mathematical optimization framework to, e.g., reduce costs or maximize earnings from flexibility. An optimization model can be used to obtain a simple policy for how to shift consumption in time and how to provide flexibility for mFRR.



%Our contribution is two-fold. \blue{First,} we show how to model mFRR reservation, bidding, and activation using a novel two-stage stochastic program and 2) we investigate and discuss the incentives to utilize demand-side flexibility for mFRR versus load shifting which is interesting from both a power system- and consumer perspective. % For example, if the consumers are incentivized to load shift instead of providing mFRR, the power system will not necessarily benefit, consumers will only benefit in the short-term.

\IEEEpubidadjcol

\vspace{-1mm}
\subsection{Dilemma: Load Shifting Versus mFRR}
%
For power consumers who are actively looking into cost saving solutions, it is important to estimate the monetary benefit of flexibility. In Nord Pool, spot prices are announced at 2pm for the next day, so supermarkets who purchase power through a retailer at  spot prices, have a chance to shift their consumption plan for the next day, accordingly. From their perspective, it is natural to first look into load shifting as the most obvious and straightforward way to utilize flexibility. Here, load shifting simply refers to shifting consumption to another point in time, typically when spot prices are lower.

Nevertheless, load shifting, although might be attractive individually, is not necessarily in  favor of the power system.
If a significant number of consumers start to act similarly, it could be detrimental to the power system by moving the peak consumption to a different time of the day. In this case, spot prices are no longer reflecting supply-demand equilibrium, and the demand distribution over the day.
%In fact, if a significant amount of consumers move a large part of their consumption to a different time of the day, that could be detrimental to the power system by creating new and even more pronounced peaks.
Intriguingly, many industrial and residential consumers have already started shifting demand to some degree, as a response to the energy crisis.
As a result, load shifting might provide short-term monetary benefit, but it is not necessarily helping the power system in the long run.

%Instead, consumers could opt for offering ancillary services, for which they are paid for providing flexibility through an \textit{aggregator}.
As an alternative option which is certainly in favor of the power system, consumers could choose to provide flexibility by offering ancillary services through an \textit{aggregator}, for which they would be compensated. Here, we focus on frequency-supporting services, and in particular mFRR, and left a potential extension by considering more services for the future work. This service is a slow-responding reserve used to stabilize frequency in the power system after fast frequency reserves are depleted. The market for mFRR is usually operated by the national Transmission System Operator (TSO), which is Energinet in Denmark. Note that mFRR resembles load shifting in the sense that it is the largest energy reserve operated by the TSO, and can be provided by consumers via shifting their consumption in time. Likewise, generators provide mFRR by increasing their power generation.

In order for a supermarket freezer to deliver mFRR, it must be part of a larger portfolio with potentially many small assets to meet the minimum bid size requirement for entering the mFRR market. In Denmark, this minimum bid size is currently 5 MW, but it is expected to reduce to 1 MW in 2024. Furthermore, there might be additional synergy effects such as increased operational flexibility in the control response, rebound, and temperature deviations \cite{koch2011modeling}. 

For simplicity, we assume that one freezer delivering mFRR can be scaled to many freezers in the same way. Therefore, 
%(and likewise for load shifting).
demand response for consumers participating in the mFRR market versus load shifting will  look similar. From the perspective of consumers, it is of great interest to explore the monetary benefit of mFRR versus load shifting. Similarly, it is important for TSOs to understand the incentive of consumers to provide mFRR services opposed to simply shifting load.

\vspace{-1mm}
\subsection{Research questions and literature review}
%
This work aims to investigate and answer the following research questions: (\textit{i}) how can the flexibility of a supermarket freezer be characterized? (\textit{ii}) how much monetary benefit can be earned by the mFRR provision opposed to load shifting? and finally (\textit{iii})
what are the pros and cons of participating in the mFRR market versus load shifting? Some of these questions have been  looked at  in the literature, so we provide a  literature review to pinpoint where new work is required, and how our work differs from the current literature.



%Demand-side flexibility has been extensively studied.
The approach taken in each study in the literature is very much dependent on the perspective from which flexibility is utilized. Often, full knowledge and control of all assets is assumed. A prevalent implicit assumption is that the aggregator and the trading entity are the same, with an exclusive business relationship to flexible consumers \cite{gade2022ecosystem}. The incentives and investments required for delivering flexibility for the flexible consumer are often overlooked as well. Many also assume an idealized market mechanism or simply propose a new mechanism for trading flexibility.

In \cite{petersen2012eso2} and \cite{pedersen2013direct}, a complete white-box model of a supermarket refrigeration system is presented and validated against real-life data. It is also shown how such a system can provide demand response. This work serves as a  benchmark for any grey-box model of a supermarket refrigeration system. However, such an approach is hard to scale and requires complete knowledge of every refrigeration system. In \cite{pedersen2016improving}, a second-order model is used to model the food and air temperature in a freezer in an experimental setting. It is shown how food temperature has much slower dynamics than air temperature. In \cite{hao2014aggregate}, a simple first-order virtual battery model for a Thermostatically Controlled Load (TCL) is presented. This model also constitutes the starting point for modeling temperature dynamics in this work, as it has an intuitive interpretation. A similar bucket model is introduced in \cite{petersen2013taxonomy}.

%In \cite{sossan2016grey}, the grey-box modeling approach to a refrigeration system is described in detail from formulation to validation. They use the grey-box model for model pblueictive control (MPC) for load shifting. In \cite{o2013modeling}, an ARMAX time-series model is used to characterize the temperature development of supermarket freezers and refrigerators. It is validated on one-step pblueiction errors, and the authors also note how it is able to roughly simulate the system over a longer period as well. This classical time-series approach provides an alternative to the state-space approach. The authors show how MPC with three different objectives can be used to optimize the flexibility in the system using the ARMAX model.

In \cite{de2019leveraging}, a Mixed-Integer Linear Programming (MILP) problem  is developed to  address up- and down-regulation hours from a baseline consumption. It is assumed that the energy for down-regulation is equal to the energy not consumed when up-regulating. A similar approach is taken in our work, except we will define the use of down-regulation until the temperature state is (approximately) back to its setpoint.

In \cite{schaperow2019simulation}, \cite{chanpiwat2020using}, \cite{moglen2020optimal}, and \cite{moglen2020optimal}, residential air condition units are modeled using up- and down-regulation blocks characterizing flexibility. The blocks are obtained from grey-box models of households \cite{siemann2013performance}. The authors show how such a block formulation can be solved using exact and stochastic dynamic programming in the context of peak shaving demand for a utility in the US. The demand-side flexibility then functions as a hedge towards extreme electricity prices. Although such a use case has not been relevant in Denmark yet, the estimation of flexibility blocks is a novel idea as it avoids having to explicitly integrate a physical model into an optimization problem. However, a tradeoff is that the curse of dimensionality quickly makes dynamic programming computationally intractable for a large portfolio of heterogeneous demand-side assets. Furthermore, it might be difficult to assure the Markov property \cite{MarkovProperty} when reformulating the problem even slightly. Flexibility blocks are also used in \cite{bobo2018offering} for an offering strategy. Here, the flexibility blocks are derived from measurements of residential appliances in the Ecogrid 2.0 project \cite{ecogrid}.

In \cite{biegel2013information} and \cite{BiegelConstractingFlexServices}, it is described how pre-defined asset characteristics allow aggregators to utilize and contract their flexibility through an interface. This assumes that such characteristics are fully known beforehand. However, this type of information flexibility interface is still useful for aggregators when engaging with consumers with well-known assets at the point where aggregation and penetration of demand-side flexibility is a mature and prevalent business.

In \cite{biegel2013electricity}, a portfolio of residential heat pumps is modeled using a linear first-order model. Individual units are lumped together as essentially one big, aggregated heat pump as described in \cite{biegel2013lumped}. The first-order model is a grey-box model incorporating the physics, i.e., the temperature dynamics in a household. Such an approach will be used in this work as well. The authors also consider how the portfolio of heat pumps can be used for load shifting and real-time bidding in the balancing market, but do not consider mFRR service provision. However, they assume that all the assets can be controlled continuously in a  model predictive control  setting. Temperature deviations are minimized directly in the objective function using the integrated error of total deviations. This introduces a trade-off parameter in the objective function that must be tuned to weigh the economic value versus the temperature deviation compared to the setpoint.

\vspace{-1mm}
\subsection{Our contribution and outline}
%
To the best of our knowledge, no work has investigated the monetary value and incentives for providing mFRR services versus load shifting, although both have been studied extensively. This work aims to fill that gap by taking a holistic view on the incentives from the flexible consumer's perspective. Furthermore, when stating the objective functions for mFRR, most studies have assumed a simplified market structure, and neglect the fact that the aggregator and the consumer are not necessarily the same entity. We discuss the consequences of this assumption in relation to the market structure in Denmark \cite{gade2022ecosystem}, and in the context of mFRR and load shifting.

We provide a realistic model formulation of the mFRR bidding, taking into account the sequence of decisions needed to be made. We propose two solution strategies for mFRR bidding: (\textit{i}) A simple one learned using the Alternating Direction Method of Multipliers (ADMM) on 2021 price data which can be applied for all of 2022, and (\textit{ii}) A dynamic one which computes a new policy every day by looking at spot prices for the past five days. Real data of a supermarket freezer  in Denmark is used.


The rest of the paper is organized as follows. Section \ref{sec:monetizing_flex} describes modeling a supermarket freezer as a TCL. Thereafter, mFRR and load shifting are described in detail and their objective functions are stated. Section \ref{sec:OptimizationModel} presents the proposed stochastic optimization problem. 
%Furthermore, the two solution strategies for solving the optimization problem for mFRR are introduced. 
Section \ref{sec:results} provides  results for a case study. Section \ref{sec:conclusion} concludes the paper. Finally, an Appendix provides the full model formulation. 



\section{Describing flexibility from TCLs}\label{sec:monetizing_flex}

There are several ways to monetize flexibility from TCLs. In this section, we focus on mFRR and load shifting. First, we describe how to mathematically model a TCL as a flexible resource. Second, we describe how to monetize the flexibility from TCLs for mFRR and load shifting, and provide the objective functions in both cases. For mFRR, the objective function includes all costs and revenues for the BRP, while for load shifting, the objective function only includes the flexible demand's perspective. This approach explicitly shows the situation where flexible demand is activated without including the BRP, as this is more realistic.

\subsection{Modelling TCL as a flexible resource}

TCLs are characterized by being controlled such that the temperature is kept at a specified setpoint. Examples include heat pumps, freezers, air condition units, ovens, etc. They are widely believed to constitute an important part of demand-side flexibility due to the inherent thermal inertia of such temperature-driven systems \cite{hao2014aggregate}.

In this paper, we focus on freezers, which are a very common type of TCLs. Specifically, we focus on a single freezer display in a Danish supermarket. Freezers are characterized by a large thermal inertia due to the frozen food, which makes them suitable for flexibility. On the other hand, there is a risk of food degradation when utilizing flexibility. Therefore, it is important to model the temperature dynamics in the freezer for a realistic and risk-aware estimation of its flexibility.

The rest of the section is organized as follows. First, we visualize the measurements from a real supermarket freezer. Second, we introduce a second-order grey-box model that characterizes the supermarket freezer. Third, we validate the second-order model and show how it can be used to simulate demand response from a freezer.

\subsubsection{Supermarket freezer description}

%In this paper, data from a single freezer operating in a large Danish supermarket is used as a case study. In Figure \ref{fig:chunk}, the air temperature and opening degree of the freezer is shown together with the electric power of the variable-speed compressor rack for a single day. All values correspond to 15-min averages. Temperature fluctuates around its setpoint at -18 $^{\circ}$C with the exception of hour 7-8, where defrosting is scheduled. While defrosting, a heating element is briefly turned on, and the expansion valve is closed such that the flow of refrigerant stops. Afterwards, while recovering the temperature, the expansion valve is fully opened. The power consumption of the compressor rack, scaled to one freezer, is shown in the bottom plot. Power consumption is highest during opening hours, and it is lowest during closing hours. During opening hours, food is being replaced and customers open the display case constantly. Furthermore, most supermarkets put additional insulation on the display cases during closing hours which reduces thermal losses. For these reasons, there are effectively two regimes for a supermarket freezer plus a short defrosting regime.

In this paper, data from a single freezer operating in a large Danish supermarket is used as a case study.
In Figure \ref{fig:chunk}, the top plot shows 15-min average air temperature of the freezer, while the middle plot shows the opening degree of the valve.
Temperature fluctuates around its setpoint at -18 $^{\circ}$C with the exception of hour 7-8, where defrosting is scheduled.
While defrosting, a heating element is briefly turned on, and the expansion valve is closed such that the flow of refrigerant stops. Afterwards, while recovering the temperature, the expansion valve is fully opened.
The electric power of the variable-speed compressor rack, scaled to one freezer\footnote{Since temperature dynamics are similar for all freezers, homogeneity is assumed. Hence, equal consumption is assumed for all freezers.}, is shown in the bottom plot.
Consumption is highest during opening hours, and it is lowest during closing hours.
During opening hours, food is being replaced and customers open the display case constantly.
Furthermore, most supermarkets put additional insulation on the display cases during closing hours which reduces thermal losses.
For these reasons, there are effectively two regimes for a supermarket freezer plus a short defrosting regime.
\begin{figure}[!t]
    \centering
    \includegraphics[width=\columnwidth]{../figures/tmp_od_Pt.png}
    \caption{\textbf{Top}: temperature of a single freezer in a supermarket. \textbf{Middle}: opening degree of the freezer expansion valve. \textbf{Bottom}: electric power of the compressor rack feeding a single freezer.}
    \label{fig:chunk}
\end{figure}

\subsubsection{Thermal modelling of freezer}

In \cite{hao2014aggregate}, it is described how a simple TCL model can be made. We extend it to a second-order state-space model that accounts for the thermal mass of the food, which essentially provides the flexibility in freezers:

\begin{subequations}\label{eq:2ndFreezerStateSpace}
    \begin{align}
        T^{f}_{t+1} & = T^{f}_{t} + dt \cdot \frac{1}{C^f}\left(\frac{1}{R^{cf}} (T^{c}_{t} - T^{f}_{t}) \right)                                                                              \\
        T^{c}_{t+1} & = T^{c}_t + dt \cdot \frac{1}{C^c}\Bigl(\frac{1}{R^{cf}} (T^{f}_t - T^{c}_t) + \frac{1}{R^{ci}_{t}} (T^{i}_t - T^{c}_t)                                          \notag \\ & \mspace{50mu} - \eta \cdot OD_t P_t \Bigr) + \epsilon \mathbbm{1}^{\text{df}}_{t}
    \end{align}
\end{subequations}

Here, $T^c$ is the air temperature in the freezer which is measured, and $T^f$ is the food temperature which is a latent, unobserved state. It is essentially a low-pass filter of the air temperature in the freezer with time constant $\tau = C^f R^{cf}$. $C^f$ and $C^c$ are the thermal capacitance of the food and air in the freezer, respectively. $R^{cf}$ and $R^{ci}$ are the thermal resistance between food and air in the freezer, and air and indoor temperature, respectively. Furthermore, $\epsilon$ represents the temperature change when defrosting and $\mathbbm{1}^{\text{df}}_{t}$ is the indicator function for when defrosting happens. $R^{ci}$ is either one of two values, $R^{ci, \text{day}}$ and $R^{ci, \text{night}}$, to capture the differences between opening hours (6 am to 10 pm) and closing hours (10 pm to 6 am). The opening degree, $OD_t$, and power $P_t$, are exogenous inputs as the opening degree is assumed to be fixed during a demand-response event, while only $P_t$ is controllable. $\eta$ is the compressor efficiency which resembles the coefficient of performance (COP). The model is discretized with a time step of 15 minutes, i.e. $dt = 0.25$ hours.

\subsubsection{Model validation}

Using the R library CTSM-R \cite{juhl2016ctsmr}, the parameters in (\ref{eq:2ndFreezerStateSpace}) have been estimated as shown in Table \ref{tab:parameter_estimates}. Notice that the thermal capacitance of the air in the freezer is significantly smaller than the thermal capacitance of the food, indicating that that the food temperature changes comparatively slower. The thermal resistance between the food and air inside the freezer, $R^{cf}$, is also significantly smaller than the thermal resistance between the air in the freezer and the indoor temperature in the supermarket, $R^{ci}$, both during the day and night. This makes sense as the lid acts as a physical barrier insulating the freezer. Furthermore, the thermal resistance to the indoor air temperature is higher during the night, which means that less power is needed, as seen in Figure \ref{fig:chunk}.

\begin{table}[!t]
    \caption{Parameter Estimates of (\ref{eq:2ndFreezerStateSpace}).}
    \label{tab:parameter_estimates}
    \centering
    \begin{tabular}[b]{|l|l|l|}
        \hline
        Parameter              & Value & Unit            \\ \hhline{|=|=|=|}
        $C^f$                  & 6.552 & kWh/$^{\circ}$C \\
        $C^c$                  & 0.077 & kWh/$^{\circ}$C \\
        $R^{cf}$               & 5.010 & $^{\circ}$C/kW  \\
        $R^{ci, \text{day}}$   & 41.05 & $^{\circ}$C/kW  \\
        $R^{ci, \text{night}}$ & 61.25 & $^{\circ}$C/kW  \\
        $\eta$                 & 1.561 &                 \\
        $\epsilon$             & 3.372 & $^{\circ}$C/h   \\ \hline
    \end{tabular}
\end{table}


The one-step residuals for the air temperature should ideally resemble white noise in order for a model to capture all dynamics seen in the data \cite{madsen2007time}. Figure \ref{fig:2ndFreezerModelValidation} shows the auto-correlation and cumulative periodogram of the residuals. The autocorrelation shows two significant lags for lag two and seven, but looks good otherwise. Likewise, in the periodogram it seems the model is able to capture most dynamics at all frequencies.

\begin{figure}[!t]
    \centering
    \includegraphics[width=\columnwidth]{../figures/2ndFreezerModelValidation.png}
    \caption{ Validation of the state-space model in (\ref{eq:2ndFreezerStateSpace}). \textbf{Left}: auto-correlation function of the model residuals. \textbf{Right}: cumulated periodogram of the residuals.}
    \label{fig:2ndFreezerModelValidation}
\end{figure}

Since there are only 96 time-steps, the defrosting period can result in relatively large residuals as it is difficult to capture such fast, transient dynamics. Since up-regulation is not allowed during defrosting, the residuals are not too important during that period. To decrease their effect in the parameter estimation procedure, the term $ \epsilon \mathbbm{1}^{\text{df}}_{t}$ was added to (\ref{eq:2ndFreezerStateSpace}) (see implementation details in code \cite{}).

Furthermore, Figure \ref{fig:2ndFreezerModelSimulation} (left) shows a 24-hour simulation of model (\ref{eq:2ndFreezerStateSpace}). It is seen that the simulation is very reasonable and closely follows the measured air temperature although a slight bias is seen as the simulated temperature is a bit higher. Such a visual validation is important because the model will be embedded in an optimization model (cf. Section \ref{sec:OptimizationModel}).

It is also important to note that ideally the validation of (\ref{eq:2ndFreezerStateSpace}) would also include real measurements from the air and food temperature in a freezer during demand response events. However, by adhering to the fundamental physics governing the temperature dynamics as shown, the model is trusted to be accurate during demand response events too. It brings an intuitive interpretation which can be used to understand the consequence to the temperature during a demand response event.

In Figure \ref{fig:2ndFreezerModelSimulation} (right), such a simulated example of a demand response event is shown. It can clearly be seen how the air temperature increases when the power is turned off, and how it decreases when the power is turned back on. The food temperature is much more stable and only changes slightly, as expected. The rebound occurs until the food temperature is back to its normal value.

\begin{figure}[!t]
    \centering
    \includegraphics[width=\columnwidth]{../figures/2ndFreezerModelSimulation.png}
    \caption{ \textbf{Left}: Simulation of (\ref{eq:2ndFreezerStateSpace}) using the parameters in Table \ref{tab:parameter_estimates}. \textbf{Right}: Simulation where power is turned off for two hours with a subsequent rebound at the nominal power until the food temperature is back to its normal value.}
    \label{fig:2ndFreezerModelSimulation}
\end{figure}


\subsection{mFRR}\label{sec:mFRR}

Figure \ref{fig:timeline_mfrr} shows the timeline of the mFRR market in Denmark.\footnote{There is only a market for up-regulation.} One should note though, that BRPs can choose \textit{not} to bid in the reserve market and only bid real-time up and down-regulation. In this work, we consider the case where flexible demand delivers reservation through a BRP since payment is received for both reservation and activation. All prices considered are for DK2.\footnote{Denmark has two synchronous areas: DK1 (western part) and DK2 (eastern part).}

First, BRPs can bid reserve capacities in each hour, $p_{h}^{r,\uparrow}$ $\forall{h} \in \{1, \ldots 24 \}$, in the market for the next day, $D$. If accepted, they receive the reservation price, $\lambda_{h}^{r,\uparrow}$. This happens \textit{before} the day-ahead market clearing for which the BRPs buy energy for their expected demand, $P_{h}^{\text{Base}}$, at the spot price, $\lambda_{h}^{s}$. After that, a regulating power bid, $\lambda_{h}^{\text{bid}}$, must be submitted for each hour in $D$ where $p_{h}^{r,\uparrow} > 0$ \cite{energinet:Systemydelser}. In real-time, the reserves are activated if the following conditions hold, i.e., if there is a reservation and if the balancing price is higher than both the spot price and the bid: $\{p_{h}^{r,\uparrow} > 0 \land \lambda_{h}^{\text{bid}} <  \lambda_{h}^{b} \land \lambda_{h}^{b} > \lambda_{h}^{s} \}$

% \begin{figure}[!t]
%     \centering
%     \includegraphics[width=\columnwidth]{../figures/timeline_mfrr.png}
%     \caption{Timeline of the Danish mFRR market.}
%     \label{fig:timeline_mfrr}
% \end{figure}


\begin{figure}[!t]
    \centering
    \includestandalone[width=\columnwidth]{../figures/timeline_mfrr_tikz}
    \caption{Timeline of the Danish mFRR market.}
    \label{fig:timeline_mfrr}
\end{figure}


If the conditions are met, the BRP receives the balancing price times their actual up-regulation, $p_{h}^{b,\uparrow}$. The BRP also incurs an additional cost due to any subsequent rebound. Furthermore, the BRP incurs a penalty, $s_{h} = \text{max}\{0, p_{h}^{r,\uparrow} - p_{h}^{b,\uparrow}$\}, if they don't deliver their promised reserve.\footnote{In reality, $p_{h}^{b,\uparrow}$ is determined by the TSO, but it is assumed here that a bid is activated in its entirety.}

A suitable objective function for a BRP delivering mFRR up-regulation for one day is therefore:

\begin{align}\label{eq:mFRRObjective}
     & \underbrace{C(\text{cost})}_{\text{mFRR}} = - \underbrace{\sum_{h=1}^{24} \lambda^{s}_{h}P^{\text{Base}}_{h}}_{\textrm{Energy cost}} + \underbrace{\sum_{h=1}^{24}\lambda_{h}^{r} p^{r, \uparrow}_{h}}_{\textrm{Reservation payment}}  \notag \\ & \quad \quad + \underbrace{\sum_{h=1}^{24}  \lambda_{h}^{b} p^{b,\uparrow}_{h}}_{\textrm{Activation payment}} - \underbrace{\sum_{h=1}^{24}  \lambda_{h}^{b} p^{b,\downarrow}_{h}}_{\textrm{Rebound cost}} - \underbrace{ \sum_{h=1}^{24}  \lambda^{p}s_{h}}_{\textrm{Penalty cost}}
\end{align}


\subsection{Load shifting}

Another option for utilizing flexibility is to shift the load to a different time according to the spot prices which are known already 12-36 hours in advance. Then it is simply a matter of consuming in low-price hours and not in high-price hours.

For a TCL, there are additional constraints to how energy can be shifted also due to the rebound. First, there can be temperature constraints which will result in less energy being shifted. Second, the rebound must happen immediately after reducing power consumption (otherwise, the temperature deviation becomes too big for too long).

The savings from load shifting are directly proportional to the volume and price difference between the baseline load and shifted load as given by:

\begin{equation}\label{eq:load_shifting_savings}
    \text{Load shifting savings} = \sum_{h=1}^{24} \lambda^{s}_{h} p^{\text{Base}}_{h} - \lambda^{s}_{h} p^{\prime}_{h}
\end{equation}

where $p_{h}$ is the power profile during load shifting, $p^{\text{Base}}_{h}$ is the anticipated baseline power, and $\lambda^{s}_{h}$ is the spot price.

However, since the load shifting action only occurs \textit{after} the day-ahead market clearing (cf. Figure \ref{fig:timeline_mfrr}), the BRP has already bought $\lambda^{s}_{h} p^{\text{Base}}_{h}$ and any deviation results in an imbalance for the BRP. In this work, we look at the case where flexible demand acts selfishly and excludes the BRP from its load shifting action. Therefore, the objective function for the flexible demand is simply:

\begin{equation}\label{eq:LoadShiftingObjective}
    \underbrace{C(\text{cost})}_{\text{load shifting}} = \sum_{h=1}^{24} \lambda_{h}^{s} p_{h}
\end{equation}


\section{Optimization model and solution strategy}\label{sec:OptimizationModel}

In this section, the optimization problem is presented. First, the time sequence of decisions for participation in mFRR is presented. Second, we explain how scenarios for price data are generated for learning a good policy in-sample (IS) and for out-of-sample (OOS) evaluation. Here, two strategies are introduced: 1) A simple one looking only at 2021 price data and 2) A dynamic one looking at the past five days of spot prices. Third, we present the compact model formulation. Fourth, we present how the mFRR bidding policy is implemented. Lastly, we show how scenario decomposition with ADMM is used to solve the first strategy with up to 250 scenarios.

\subsection{Time sequence for decision making}

Figure \ref{fig:timeline_mfrr_variables} shows the stages for making decisions in the mFRR market. First, a reservation bid $p_{h}^{r,\uparrow}$ is submitted. For any reservation bid accepted, a regulating power bid must be submitted, $\lambda_{h,\omega}^{\text{bid}}$. The reservation and bid are the \textit{first-stage} decisions. The set $\Gamma_{\omega}$ contains all real-time variables in the optimization problem and are the \textit{second-stage} decisions. They control the real-time power, auxiliary variables for identifying up- and down-regulation\footnote{Down-regulation refers to the rebound action in this context.}, temperature dynamics, and when to deliver up-regulation according to the bid and prices (see Appendix A for a detailed description). Index $\omega$ specifies a scenario.
% \begin{figure}[!t]\label{fig:timeline_mfrr_variables}
%     \centering
%     \includegraphics[width=\columnwidth]{../figures/timeline_mfrr_variables.png}
%     \caption{Variables related to mFRR up-regulation decisions. The asterisk indicates the first-stage decision.}
% \end{figure}

\begin{figure}[!t]
    \centering
    \includestandalone[width=\columnwidth]{../figures/timeline_mfrr_variables_tikz}
    \caption{Variables related to mFRR up-regulation decisions. The asterisk indicates the first-stage decision, and \\ $\Gamma_{\omega} = \{ \bm{p}^{r,\uparrow}, \bm{\lambda}_{\omega}^{\text{bid}}, \bm{p}_{\omega}^{b,\uparrow}, \bm{p}_{\omega}^{b,\downarrow}, \bm{s}_{\omega}, \bm{T}_{\omega}^{c}, \bm{T}_{\omega}^{f}, \bm{T}_{\omega}^{c, B}, \bm{T}_{\omega}^{f,B}, \bm{\phi}_{\omega}, \bm{g}_{\omega} \}$ is the set of second-stage decision variables.}
    \label{fig:timeline_mfrr_variables}
\end{figure}

\subsection{Scenario generation}\label{sec:scenario_generation}

To learn a policy from the optimization model, i.e., $p_{h}^{r,\uparrow}$ and $\lambda_{h,\omega}^{\text{bid}}$, we generate scenarios for price data using two different strategies (where one scenario corresponds to one day):

\begin{enumerate}
    \item Policy learned IS on 2021 price data with either 1, 5, 10, 20, 30, 40, 50, 100, or 250 scenarios
    \item Policy learned using a lookback of five days for spot prices
\end{enumerate}

In both cases, balancing prices, $\lambda_{h}^{b}$, are sampled from days where up-regulation happened $0 \ldots 24$ hours with equal probability. In this way, days where up-regulation happened are essentially up-sampled. The reason for this strategy is that the model learns more when up-regulation happens than when it does not.

For spot prices, the lookback strategy uses the spot price from the past five days to take advantage of any autocorrelation, whereas the first strategy samples spot prices from 2021. For both strategies, evaluation is done OOS on unseen 2022 price data.

For load shifting, the policy is simply to solve the optimization problem for the next day since the day-ahead market clearing happens in advance.

\subsection{Compact model formulation}

The compact model formulation is presented in Problem (\ref{P1:compact_model}). The objective function is the sum of the revenue from the reservation bid, i.e., first-stage decisions, and the \textit{expected} revenue and costs from the regulating power bid and subsequent rebound and penalties. Constraint (\ref{P1:eq2}) incorporates activation of the bid, and constraint (\ref{P1:eq3}) incorporates the temperature dynamics in the freezer.


\begin{subequations}\label{P1:compact_model}
    \begin{align}
        \underset{\bm{p}^{r,\uparrow}, \bm{\lambda}_{\omega}^{\text{bid}}, \bm{\Gamma}_{\omega}}{\textrm{max}} \quad & f(\bm{p}^{r,\uparrow}) + \sum_{\omega} \pi_{\omega} g(\bm{\Gamma}_{\omega}) \label{P1:eq1}
        \\
        s.t \quad                                                                                                    & h(\bm{p}^{r,\uparrow}, \bm{\lambda}_{omega}^{\text{bid}}, \bm{\Gamma}_{\omega}) \leq 0, \quad \forall{\omega}\label{P1:eq2}                                                                     \\
        \quad                                                                                                        & \text{State-space model in } (\ref{eq:2ndFreezerStateSpace}), \quad \forall{\omega} \label{P1:eq3}
        \\
        \quad                                                                                                        & \Bigl( \bm{p}^{r,\uparrow}, \bm{\lambda}_{\omega}^{\text{bid}}, \bm{p}_{\omega}^{b,\uparrow}, \bm{p}_{\omega}^{b,\downarrow}, \bm{s}_{\omega}, \bm{T}_{\omega}^{c}, \bm{T}_{\omega}^{f}, \notag \\ \quad & \quad \bm{T}_{\omega}^{c, B}, \bm{T}_{\omega}^{f,B}, \bm{\phi}_{\omega}, \bm{g}_{\omega} \Bigr) \in \mathbb{R}^{n}  \label{P1:eq4}
        \\
        \quad                                                                                                        & \bm{u}_{\omega}, \bm{z}_{\omega}, \bm{y}_{\omega} \in \{0,1\}  \label{P1:eq5}
    \end{align}
\end{subequations}

For mFRR, the objective function in (\ref{eq:mFRRObjective}) corresponds to $f$ and $g$ in (\ref{P1:eq1}). For load shifting, the optimization problem is simply obtained by removing bid constraints and replacing the (\ref{P1:eq1}) with (\ref{eq:LoadShiftingObjective}), and then solving the optimization problem for the next day.


\subsection{mFRR bidding implementation}\label{sec:mFRR_bidding_implementation}

To solve Problem (\ref{P1:compact_model}), we first need to specify a bidding policy that can readily be used OOS. We do so by choosing an affine bidding policy. Afterwards, it is shown how the bidding policy is implemented using McCormick relaxation.

\subsubsection{Affine bidding policy}

A bidding policy needs to be easy to follow OOS for the trader. We choose an affine bidding policy, i.e., a linear function of the spot price. The bidding policy is given by:

\begin{subequations}\label{eq:affine_policy}
    \begin{align}
        \lambda^{bid}_{h,\omega}           & = \alpha \lambda_{h,\omega}^{s,\text{diff}} + \beta + \lambda_{h,\omega}^{s}, \quad \forall{\omega}, \forall{h} \in \{1\ldots23\}\label{affine_policy:1} \\
        \lambda_{h,\omega}^{s,\text{diff}} & = \lambda_{h+1,\omega}^{s} - \lambda_{h,\omega}^{s}, \quad \forall{\omega}, \forall{h} \in \{1\ldots23\}\label{affine_policy:1_2}                        \\
        \alpha                             & \geq 0\label{affine_policy:2}                                                                                                                            \\
        \beta                              & \geq 0\label{affine_policy:3}
    \end{align}
\end{subequations}

The idea in (\ref{eq:affine_policy}) is that the spot price differential approximates a reasonable policy since a large price differential in hour $h$ means that the spot price at time $h+1$ is much higher. Hence, a large bid at time $h$ is desirable such that the probability of being activated in hour $h$ and rebounding in an expensive hour $h+1$ is low (cf. Section \ref{sec:mFRR}).

Variables $\alpha$ and $\beta$ is then learned IS and fixed for OOS evaluation. After the day-ahead market clearing, (\ref{eq:affine_policy}) can easily be used to specify bids for the next day.

\subsubsection{McCormick relaxation}\label{sec:mccormick}

As explained in Section \ref{sec:mFRR}, activation of mFRR reservation only happens when certain price conditions are met. This is formalized in the following constraint:

\begin{equation}\label{eq:bid_constraint}
    p^{b, \uparrow}_{h,\omega} + s_{h,\omega} \geq p^{r,\uparrow}_{h} \cdot \mathbbm{1}_{h,\omega}^{(\lambda^{\text{bid}}_{h} < \lambda^{b}_{h,\omega} \land \lambda^{b}_{h,\omega} > \lambda^{s}_{h})}, \quad \forall{h,\omega}
\end{equation}

Eq. (\ref{eq:bid_constraint}) shows how real-time up-regulation plus a slack variable must be greater than or equal to the reservation if the bid is lower than the balancing price and if up-regulation is needed in hour $h$. It is a bi-linear constraint so McCormick relaxation \cite{mccormick1976computability} is used to convert (\ref{eq:bid_constraint}) to a linear constraint by introducing auxillary variables, $\phi_{h,\omega}$ and $g_{h,\omega}$:

\begin{subequations}\label{eq:bid_constraint_relaxed}
    \begin{align}
        \lambda_{h,\omega}^{b} - \lambda_{h}^{s} \geq \lambda_{h,\omega}^{bid} - M \cdot (1 - g_{h,\omega}) , \quad                                   & \forall{h,\omega}             \label{con_bid:subeq1} \\
        \lambda_{h,\omega}^{bid} \geq \lambda_{h,\omega}^{b} - \lambda_{h}^{s} - M \cdot (1 - g_{h,\omega}) , \quad                                   & \forall{h,\omega}             \label{con_bid:subeq2} \\
        p^{b, \uparrow}_{h,\omega} \leq \phi_{h,\omega} \cdot \mathbbm{1}_{h,\omega}^{\lambda^{b}_{h,\omega} > \lambda^{s}_{h}}, \quad                & \forall{{h,\omega}}           \label{con_bid:subeq3} \\
        p^{b, \uparrow}_{h,\omega} + s_{h,\omega} \geq \phi_{h,\omega} \cdot \mathbbm{1}_{h,\omega}^{\lambda^{b}_{h,\omega} > \lambda^{s}_{h}}, \quad & \forall{{h,\omega}}           \label{con_bid:subeq4} \\
        -g_{h,\omega} \cdot M \leq \phi_{h,\omega}, \quad                                                                                             & \forall{h,\omega}             \label{con_bid:subeq5} \\
        \phi_{h,\omega} \leq g_{h,\omega} \cdot M, \quad                                                                                              & \forall{h,\omega}             \label{con_bid:subeq6} \\
        -(1 - g_{h,\omega}) \cdot M \leq \phi_{h,\omega} - p^{r,\uparrow}_{h}, \quad                                                                  & \forall{h,\omega}             \label{con_bid:subeq7} \\
        \phi_{h,\omega} - p^{r,\uparrow}_{h} \leq (1 - g_{h,\omega}) \cdot M, \quad                                                                   & \forall{h,\omega}             \label{con_bid:subeq8}
        % \lambda_{h,\omega}^{bid} \leq \lambda^{Max} \label{con_bid:subeq9}
    \end{align}
\end{subequations}

Constraints (\ref{con_bid:subeq1}-\ref{con_bid:subeq2}) ensures that $g_{h,\omega} = 1$ when the balancing price minus the spot price is larger than our bid, $\lambda^{bid}_{h, \omega}$, and zero otherwise. Constraints (\ref{con_bid:subeq3}-\ref{con_bid:subeq4}) sets the TCL up-regulation equal to $\phi_{h,\omega}$ (or incurs a penalty through $s_{h,\omega}$) if there is an up-regulation event in the system, i.e., if $\mathbbm{1}_{h,\omega}^{\lambda^{b}_{h,\omega} > \lambda^{s}_{h}} = 1$. Constraints (\ref{con_bid:subeq5}-\ref{con_bid:subeq6}) ensures that $\phi_{h,\omega} = 0$ when $g_{h,\omega} = 0$, i.e., when the balancing price differential is smaller than our bid. Constraints (\ref{con_bid:subeq7}-\ref{con_bid:subeq8}) ensures that $\phi_{h,\omega}$ is equal to the reservation capacity, $p^{r,\uparrow}_{h}$, whenever $g_{h,\omega} = 1$, i.e., whenever the bid is smaller than the balancing price differential.

The final model formulation is shown in its entirety in Appendix A.

\subsection{Scenario decomposition with ADMM}

When solving Problem (\ref{P1:compact_model}) using the first strategy with many scenarios, it quickly becomes computationally intractable due to the number of binaries and intertemporal constraints. To solve it, the ADMM algorithm is used which can solve a large-scale optimization problem by decomposing it into smaller subproblems \cite{boyd2011distributed}. In (\ref{P1:compact_model}), each scenario is solved as a subproblem by setting:

\begin{equation}\label{eq:non_anticipativity}
    \bm{p}^{r,\uparrow} \rightarrow \bm{p}^{r,\uparrow}_{\omega}, \quad \alpha \rightarrow \alpha_{\omega}, \quad \beta \rightarrow \beta_{\omega}
\end{equation}

The ADMM algorithm will converge by achieving consensus on first- and second-stage stage decisions in linear problems, but it is only a heuristic for MILPs \cite{hong2016convergence}.


\section{Numerical Results and Discussion}\label{sec:results}
%
We consider five models to calculate the operational (energy) cost of a single freezer, including three models where the freezer provides mFRR services, one model where the freezer shifts the load in response to spot prices, and the last model, the so-called \textit{base cost}, where the freezer neither shifts the load nor provides mFRR services. The data availability for these five models is summarized in
Table \ref{tab:price_visibility}. All source codes and relevant data are publicly shared in \cite{code}.

\vspace{-2mm}
\subsection{Load shifting vs mFRR: Which one is more appealing?}
%
\begin{table}[t]
    \caption{Data availability for each model.}
    \label{tab:price_visibility}
    \centering
    \begin{tabular}{llccc}
        \toprule
        Model         & Curve(s) in Fig. \ref{fig:cumulative_cost_comparison} & $\bm{\lambda}^{\rm{r},\uparrow}$ & $\bm{\lambda}^{\rm{s}}$ & $\bm{\lambda}^{\rm{b}}$ \\
        \midrule
        mFRR oracle   & Dashed black                                          & $\checkmark$                     & $\checkmark$            & $\checkmark$            \\
        mFRR cases    & Blue and red                                          & $\checkmark$                     & $\ast$                  & $\ast$                  \\
        Load shifting & Yellow                                                & N/A                              & $\checkmark$            & N/A                     \\
        Base cost     & Solid black                                           & N/A                              & $\checkmark$            & N/A                     \\
        \bottomrule
        \multicolumn{5}{l}{$\checkmark$: True (realized) data are available.}                                                                                        \\
        \multicolumn{5}{l}{$\ast$: Forecast data (in form of scenarios) are available.}                                                                              \\
        \multicolumn{5}{l}{N/A: Not applicable.}
    \end{tabular}
    \vspace{-3mm}
\end{table}




For all five models, Fig. \ref{fig:cumulative_cost_comparison} shows the out-of-sample cumulative operational cost of the single freezer during the first nine months of 2022. Recall that all these models have been compared fairly by an out-of-sample simulation against identical scenarios, i.e., real spot and balancing market prices from 2022.
The base cost (solid black) has only access to spot price data, and leads to the highest cost in Fig. \ref{fig:cumulative_cost_comparison}.
Blue and red curves (mFRR cases), both below the base cost curve, correspond to the cases where the freezer provides mFRR services, and uses scenarios to model spot and balancing market price uncertainties. Their difference comes from in-sample scenarios used: while the red curve uses five equiprobable scenarios coming from the most recent historical data (lookback strategy), the blue curve uses 50 equiprobable scenarios coming from 2021. The yellow curve shows the reduced cost of the freezer due to  load shifting. Finally, the dotted black curve (lowest in Fig. \ref{fig:cumulative_cost_comparison}) provides an oracle (ideal benchmark), where the freezer perfectly knows the spot and balancing market prices.

An interesting observation is that, in comparison to the base cost, load shifting (yellow curve) bring a 13.9\% cost reduction, whereas the mFRR  provision  reduces the cost by  11.6\% (red curve) and 10.1\% (blue curve). Note that the cost savings reported for red and blue curves are not far  from the mFRR oracle, showing that the scenarios used are sufficiently adequate.
While this observation might be appealing to the freezer as load shifting requires a comparatively simpler decision-making process than the mFRR provision, it is not necessarily a desirable outcome for the power system. The mFRR  provision helps the system keep the supply-demand balance, while load shifting as a response to spot prices is not necessarily helping the system, and even in the worse case, the spot prices fixed before load shifting may no longer represent the supply-demand equilibrium. This calls system operators and regulators for potential changes, e.g., market redesign, to make the mFRR  provision more attractable.



\begin{figure}[t]
    \centering
    \includegraphics[width=\columnwidth]{../figures/cumulative_cost_comparison.png}
    \caption{Out-of-sample cumulative operational cost for the freezer during the first nine months of 2022.}
    \label{fig:cumulative_cost_comparison}
    \vspace{-2mm}
\end{figure}

In the case of load shifting, it is worth mentioning that the settlement cost of the BRP is ignored --- this reflects the self-interested flexible loads.  While the flexible load reduces her cost by load shifting, the corresponding BRP may incur a settlement cost due to the resulting imbalance between the true consumption and the day-ahead schedule. In practice, the BRP would have to pay for such an imbalance or at least buy/sell energy in the intra-day market to adjust her day-ahead schedule and consider potential load shifting actions by flexible loads. However, loads are not necessarily aware of those costs for the BRP. Hence, they may have a strong incentive to exploit their flexibility for load shifting. Some larger flexible consumers, such as industrial and commercial loads, might not be exposed fully to spot prices, and for those consumers, load shifting could be less profitable. However, they may still have an incentive to change their deal with the BRP to get a full exposure.

In the case of mFRR  provision, it is   assumed that the entire revenue from reservation and activation go to the flexible load. This neglects the fact that, in practice, the BRP requires a share of the revenue. Furthermore, there might be an aggregator or a technology provider who facilitates the aggregation and communication of the flexibility. In such a case, they would also request a share of the revenue. This can potentially reduce the revenue of flexible loads by the mFRR provision.








%\begin{flushleft}
%    \begin{table}[b]
%        \caption{Average Daily OOS Costs.}
%        \label{tab:cases_compared}
%        \centering
%        \begin{tabular}{lccc}
%            \toprule
%            Name                 & \thead{mFRR w.                   \\lookback} & Load shifting & \thead{mFRR \\w. 2021} \\
%            \midrule
%            Base cost today      & 40.6         & 40.6 & 40.6 \\
%            Total cost           & 35.9         & 35 & 36.5 \\
%            Expected energy cost & 40.6         & 35 & 40.6 \\
%            Rebound cost         & 0.86          & N/A    & 0.9  \\
%            Reserve payment      & 3.2          & 0.0    & 3.4  \\
%            Act payment          & 8.1          & 0.0    & 2.2  \\
%            Penalty cost         & 5.7          & 0.0    & 0.5  \\
%            Scenarios            & -5             & 1      & 50     \\
%            ADMM                 & False          & False  & True   \\
%            \% savings           & 11.6           & 13.9   & 10.1   \\
%            \bottomrule
%        \end{tabular}
%    \end{table}
%            \vspace{-2mm}
%\end{flushleft}


%Table \ref{tab:cases_compared} breaks down the cost components. For mFRR, there is big difference between the lookback strategy and the ADMM strategy with 2021 prices. The lookback strategy earns much more from activation payments, but is also penalized more as it is not able to deliver its reservation capacity in some days. The other mFRR strategy bids more conservatively and is only rarely activated. The market rules prescribe that the full bid should be delivered, hence the lookback strategy might be too risky. On the other hand, it was assumed that the activation power should be equal to the reserved power, but the TSO determines the activation power, hence the activation power might be lower in reality which would both decrease the penalty cost and the activation revenue.

%All strategies are better than the baseline costs, i.e., not utilizing flexibility, with savings of 10-14\%. Furthermore, the mFRR strategies are not far off the theoretically best mFRR strategy as indicated by the oracle in Figure \ref{fig:cumulative_cost_comparison}.


\begin{figure}[t]
    \centering
    \includegraphics[width=\columnwidth]{../figures/spot_single_case.png}
    \caption{Example of load shifting in a representative day (an in-sample scenario). \textbf{Top}: Baseline consumption of the freezer and the power profile after load shifting. \textbf{Middle}: Air and food temperature dynamics. \textbf{Bottom}: Spot market prices.}
    \label{fig:fig_first_case}
    \vspace{-4mm}
\end{figure}




\begin{figure}[t]
    \centering
    \includegraphics[width=\columnwidth]{../figures/mFRR_single_case.png}
    \caption{Example of mFRR  provision  in a representative day (an in-sample scenario). \textbf{Top}: Reservation capacities and baseline power of the freezer. \textbf{Upper middle}: Air and food temperature dynamics. \textbf{Lower middle}: mFRR activation in this specific scenario, i.e., when $\lambda_{h}^{\rm{bid}} \leq \lambda_{h}^{\rm{b}}-\lambda_{h}^{\rm{s}}$, $\lambda_{h}^{\rm{b}} > \lambda_{h}^{\rm{s}}$, and $p^{\rm{r},\uparrow}_{h} > 0$. \textbf{Bottom}: Spot and balancing market price as well as regulating power bids in this specific scenario.}
    \label{fig:fig_second_case}
        \vspace{-4mm}
\end{figure}


\vspace{-2mm}
\subsection{Technical results}
Fig. \ref{fig:fig_first_case} shows technical results for the case of load shifting in a representative day. For given spot prices (bottom plot), it is evident that the freezer shifts the load to low-price hours, especially in the later hours of the day (top plot). However, this has a remarkable effect on the air and food temperatures with large deviations from the normal set-point (middle plot).

%, load shifting and mFRR are compared for the same day. For load shifting, it can clearly be seen how load is shifted to low-price hours, especially at the end of the day. But it also has a very significant effect on the temperature with large deviations from its normal setpoint. 

Fig. \ref{fig:fig_second_case} presents technical results for the case of mFRR  provision  in a representative day, whose spot and balancing market prices are given in the lower plot. The top plot shows the baseline power profile $P^{\text{Base}}_{h}$ and the mFRR reservation $p^{\rm{r}, \uparrow}_{h}$ sold. For a better illustration, we have plotted $P^{\text{Base}}_{h}-p^{\rm{r}, \uparrow}_{h}$, i.e., the power profile of the freezer if the full activation happens during the entire day. The third  plot of Fig. \ref{fig:fig_second_case} shows, for this specific day (as an in-sample scenario), the freezer is activated three times, i.e., in hours 6, 9, and 10. In these three hours, all conditions $\lambda_{h,\omega}^{\rm{bid}} \leq  \lambda_{h,\omega}^{\rm{b}} - \lambda_{h,\omega}^{\rm{s}}$, $ \lambda_{h,\omega}^{\rm{b}} > \lambda_{h,\omega}^{\rm{s}}$, and $p^{\rm{r},\uparrow}_{h} > 0$ hold (see the bottom plot). The rebound  happens in hours 11 to 19. The majority of rebound occurs in hour 19, as the balancing price is comparatively low. Finally, the second plot shows that the food and air temperature deviations in the freezer due to mFRR  provision  are smoother in comparison to  load shifting.


%For mFRR, the reservation is almost full for all hours during the day. The activation of reservation occurs when the bid price is lower than the balancing price which in this particular scenario only happens for three hours. The effect on the temperature is therefore much smaller than for load shifting. The model is also able to rebound smartly in hour 19 to avoid high rebound costs.

Our second main observation is as follows: Although the cost saving is higher for load shifting in comparison to mFRR  provision, it is directly proportional to the energy shifted, and therefore, the temperature deviation in the freezer. This is not the case for mFRR, as the reservation is not always activated. 



% AS ONE FIGURE
% \begin{figure*}[t]
%     \centering
%     \subfloat[]{\includegraphics[width=\columnwidth]{../figures/spot_single_case.png}%
%         \label{fig_first_case}}
%     \hfil
%     \subfloat[]{\includegraphics[width=\columnwidth]{../figures/mFRR_single_case.png}%
%         \label{fig_second_case}}
%     \caption{Comparison between load shifting and mFRR in one IS scenario. \textbf{a}: \textbf{Top}: Power profile when load shifting and baseline power of freezer. \textbf{Middle}: Air and food temperature dynamics. \textbf{Bottom}: Spot price in scenario. \textbf{b}: \textbf{Top}: Reservation capacities and baseline power of freezer. \textbf{Upper middle}: Air and food temperature dynamics. \textbf{Lower Middle}: mFRR activations in this scenarios, i.e., when $\lambda_{h}^{bid} \leq \lambda_{h}^{b}$, $\lambda_{h}^{b} > \lambda_{h}^{s}$ and $p^{r,\uparrow}_{h} > 0$. \textbf{Bottom}: Spot price, balancing price, and bid price in scenario.}
%     \label{fig:fig_sim}
% \end{figure*}

% AS TWO FIGURES





%Nevertheless, load shifting seems more appealing for a flexible consumer compared to mFRR from a monetary point of view. This finding illustrates the importance of designing attractive markets for mFRR if demand-side flexibility is to be used more widely, and perhaps also to disincentivize load shifting for flexible consumers as it could lead to system imbalances.


%\subsection{ADMM}
%Figure \ref{fig:admm_vs_normal_solution} shows the ADMM convergence to the optimal solution for five scenarios for different step sizes. For most step sizes, it converges quickly but never quite reaches the optimal solution. This is mainly because the algorithm has to achieve consensus on three variables, $p_{h,\omega}^{r,\uparrow}$, $\alpha_{\omega}$, and $\beta_{\omega}$. Experiments showed that it was converging much closer to the optimal solution when removing $\alpha_{\omega}$, and $\beta_{\omega}$.

%A large step size emphasizes the need to reach consensus whereas a small step size prioritizes the objective function in (\ref{P1:eq1}). Here, it seems a step size of $\gamma \geq 1$ is more stable and converges quickly.

%\begin{figure}[!t]
%    \centering
%    \includegraphics[width=\columnwidth]{../figures/admm_vs_normal_solution.png}
%    \caption{ADMM solution versus the optimal solution for five scenarios for different step sizes in the ADMM algorithm.}
%    \label{fig:admm_vs_normal_solution}
%\end{figure}

%Figure \ref{fig:admm_nb_scenarios_effect} shows the effect of including more scenarios in Problem (\ref{P1:compact_model}) using ADMM to solve it. For IS, good solutions are already obtained with 5-10 scenarios, and the same applies for OOS although it seems using 250 scenarios also performs well.

%The plot highlights the importance of choosing representative scenarios, especially balancing prices as they determine how much activation is needed, and therefore the bid policy.

%\begin{figure}[!t]
%    \centering
%    \includegraphics[width=\columnwidth]{../figures/admm_nb_scenarios_effect.png}
%    \caption{Effect of number of IS scenarios on OOS performance for ADMM. Both are compared to the baseline costs of the freezer.}
%    \label{fig:admm_nb_scenarios_effect}
%\end{figure}

% \subsection{Lookback}

% TODO: create plot of effect of lookback parameter.


\section{Conclusion}\label{sec:conclusion}

We investigated how a supermarket freezer can provide flexibility for mFRR and load shifting in Denmark, and explored which one provides a greater monetary incentive for a flexible load. To this end, we used actual data  from a Danish supermarket. This was done by developing a second-order grey-box model of the temperature dynamics in the freezer with the food temperature as a latent state. In the state-space form, the model was directly incorporated as constraints into a two-stage stochastic MILP problem, whose objective is to maximize the monetary value from the freezer's flexibility. Two scenario generation strategies were implemented: one with a five-day lookback strategy on the spot market prices, and the other one  based on the Danish market data for 2021. For mFRR, we used a linear policy, and then linearized the conditions for activation through  a McCormick relaxation. For computational ease, we used an ADMM-based scenario decomposition technique.  An out-of-sample evaluation was done on unseen 2022 price data. We observed that load shifting was more profitable, but had a greater impact on the air and food temperatures in the freezer as opposed to mFRR that depends on the system state and bid price for activation. 

We made a set of simplifications and assumptions that need to be relaxed for the future work. The revenue share of BRP and aggregator was not considered. This may change our finding that load shifting is more financially appealing to a flexible load. As mentioned earlier, a single freezer or supermarket must be part of a larger portfolio through an aggregator in order to participate in the mFRR market. Such an aggregated portfolio has some issues that are neglected here, such as the baseline estimation for verification of the demand response, allocation of profits within the portfolio, and an accurate capacity estimation of the whole portfolio that bids in the mFRR market. Furthermore, the European mFRR markets will change from a 60-minute resolution to a 15-minute market in the next few years \cite{MARI}. This makes it more feasible for TCLs to participate in the mFRR market, given their sensitivity to large temperature deviations.

{\appendices
\vspace{-1mm}
\section*{Appendix: MILP problem formulation}\label{appendix:A}
% \addcontentsline{toc}{section}{Appendices}
% \renewcommand{\thesubsection}{\Alph{subsection}}
%\subsection{MILP problem formulation}\label{appendix:A}

%\subsection{Final model formulation}\label{sec:final_model}

% Describe objective function for mFRR and load shifting.

% The set, $\Psi$, then contains all the optimization variables:

% \begin{align}\label{set:OptVariables}
%     \Psi = \{p_{h,\omega}, p^{\rm{r}, \uparrow}_{h}, & s_{h,\omega}, p^{\rm{b}, \uparrow}_{h,\omega}, p^{\rm{b}, \downarrow}_{h,\omega}, u^{\uparrow}_{h,\omega}, y^{\uparrow}_{h,\omega}, z^{\uparrow}_{h,\omega}, u^{\downarrow}_{h,\omega}, y^{\downarrow}_{h,\omega}, z^{\downarrow}_{h,\omega}, \notag \\ & T^{f}_{\omega, t}, T^{c}_{\omega, t}, T^{f,B}_{t}, T^{c, B}_{t}, \lambda_{h}^{bid}, \phi_{h,\omega}, g_{h,\omega}, \Delta \}
% \end{align}



%(\ref{P1:eq4})-(\ref{P1:eq5})

\begingroup
\allowdisplaybreaks
The two-stage stochastic MILP problem for the freezer to optimally bid in the mFRR market reads as
% First, the objective function is presented. Then all auxillary variables and constraints are presented. Third, constraints related to the power consumption for the freezer is shown. Fourth, the physical constraints for the temperatures are presented. Lastly, the rebound constraints are presented.
% \subsubsection{Objective function}\label{sec:objective_function}
%
\begin{subequations}\label{P2:FinalModel}
    \begin{align}
       & \underset{p_{h}^{\rm{r},\uparrow}, \alpha, \beta, \Gamma_{h,\omega}}{\text{Maximize}} \ \sum_{h=1}^{24}\lambda_{h}^{\rm{r}} p^{\rm{r}, \uparrow}_{h}+ \sum_{\omega=1}^{|\Omega|} \pi_{\omega}  \Bigl(\sum_{h=1}^{24}  \lambda_{h,\omega}^{\rm{b}} p^{\rm{b},\uparrow}_{h,\omega} - \notag                                                     \\  &  \hspace{3.5cm}\sum_{h=1}^{24}  \lambda_{h,\omega}^{\rm{b}} p^{\rm{b},\downarrow}_{h,\omega} - \sum_{h=1}^{24}  \lambda^{\rm{p}}s_{h,\omega} \Bigr) \label{P2:1} \\
       %
        & \   \text{s.t.}  \  (\ref{eq:bid_constraint_relaxed}), \ \forall{h,\omega},   \label{P2:2}    \\ 
        & \                                               \text{(\ref{P1:eq4})-(\ref{P1:eq6})}, \alpha \geq 0,   \beta \geq 0 \    \label{P2:3}    \\ 
%
& T^{\rm{f}}_{t+1,\omega} = T^{\rm{f}}_{t,\omega} +  \frac{dt}{C^{\rm{f}}R^{\rm{cf}}} (T^{\rm{c}}_{t,\omega} - T^{\rm{f}}_{t,\omega}), \ \forall{t<J-1,\omega} \label{P2:state-space-1}                                                                                         \\
& T^{\rm{c}}_{t+1,\omega} = T^{\rm{c}}_{t,\omega} +    \frac{dt}{C^{\rm{c}}}\Bigl(\frac{1}{R^{\rm{cf}}} (T^{\rm{f}}_{t,\omega} - T^{\rm{c}}_{t,\omega}) +  \notag \\ & \frac{1}{R^{\rm{ci}}} (T^{\rm{i}}_t - T^{\rm{c}}_{t,\omega}) - \eta  \  OD_t \ p_{h,\omega} \Bigr) + \epsilon \mathbbm{1}^{\rm{df}}_{t}, \notag \\ & \hspace{5.6cm} \forall{t<J-1,\omega} \label{P2:state-space-2} \\
%
& T^{\rm{f},\text{Base}}_{t+1} = T^{\rm{f},\text{Base}}_{t} +   \frac{dt}{C^{\rm{f}}R^{\rm{cf}}} (T^{\rm{c},\text{Base}}_{t} - T^{\rm{f},\text{Base}}_{t}), \notag \\ & \hspace{6cm}\forall{t<J-1} \label{P2:state-space-3}                                                                                         \\
& T^{\rm{c},\text{Base}}_{t+1} = T^{\rm{c},\text{Base}}_t +   \frac{dt}{C^{\rm{c}}}\Bigl(\frac{1}{R^{\rm{cf}}} (T^{\rm{f},\text{Base}}_t - T^{\rm{c},\text{Base}}_t) + \notag \\ & \frac{1}{R^{\rm{ci}}} (T^{\rm{i}}_t - T^{\rm{c},\text{Base}}_t) - \eta  \  OD_t \ p_{h}^{\text{Base}} \Bigr) + \epsilon \mathbbm{1}^{\rm{df}}_{t}, \notag \\ & \hspace{6cm} \forall{t<J-1} \label{P2:state-space-4} \\
%
& \ p_{h,\omega} = P^{\rm{Base}}_{h} - p^{\rm{b}, \uparrow}_{h,\omega} + p^{\rm{b}, \downarrow}_{h,\omega}, \                                                                                                  \forall{h,\omega}                                                                             \label{power:6}                                                        \\
        \                                                                                             & p^{\rm{r}, \uparrow}_h \leq P^{\rm{Base}}_h,
        \                                                                                                                                                        \forall{h}                                                                                     \label{power:7}                                                                                                                                                                                                           \\
        \                                                                                             & p^{\rm{b}, \uparrow}_{h,\omega} \leq p^{\rm{r}, \uparrow}_h \mathbbm{1}_{h,\omega}^{\lambda^{\rm{b}}_{h,\omega} > \lambda^{\rm{s}}_{h,\omega}} , \                                                                            \forall{h,\omega}                                                                             \label{power:8}                                                   \\
        \                                                                                             & p^{\rm{b}, \uparrow}_{h,\omega} \leq u_{h,\omega}^{\uparrow} \big(P^{\rm{Base}}_{h} - P^{\rm{Min}}\big) , \                                                                                                       \forall{h,\omega}                                                                             \label{power:9}                                                  \\
        \                                                                                             & p^{\rm{b}, \downarrow}_{h,\omega} \leq u^{\downarrow}_{h,\omega} \big(P^{\rm{Nom}} -P^{\rm{Base}}_{h}\big), \                                                                                              \forall{h,\omega}                                                                             \label{power:10}                                                        \\
        \                                                                                             & P^{\rm{Min}} \leq p_{h,\omega} \leq P^{\rm{Nom}}, \                                                                                                                                           \forall{h,\omega}                                                                             \label{power:11}                                                        \\
        \                                                                                             & 0 \leq s_{h,\omega} \leq P^{\rm{Base}}_{h}, \                                                                                                                                                   \forall{h,\omega}                                                                             \label{power:12}                                                        \\
        % \                                                                                             & \ p^{\rm{b}, \uparrow}_{h,\omega} + s_{h,\omega} \geq p^{\rm{r},\uparrow}_{h} \  \mathbbm{1}_{h,\omega}^{(\lambda^{\rm{bid}}_{h} < \lambda^{\rm{b}}_{h,\omega}, \lambda^{\rm{b}}_{h,\omega} > \lambda^{s}_{h})}, \                                                                                                              \forall{h,\omega} \label{power:13} \\
        \                                                                                             & p^{\rm{b}, \downarrow}_{h,\omega} \geq 0.10 \  u^{\downarrow}_{h,\omega} \big(P^{\rm{Nom}} - P^{\rm{Base}}_{h}\big), \                                                                                  \forall{h,\omega}                                                                             \label{power:14}                                                        \\
        \                                                                                             & p^{\rm{r}, \uparrow}_{h} \leq P^{\rm{Base}}_{h} \big(1 - \mathbbm{1}_{h}^{\rm{df}}\big), \                                                                                                                 \forall{h} \label{power:15}                                                                                                                                           \\
        \                                                                                             & u_{h-1,\omega}^{\uparrow} - u_{h,\omega}^{\uparrow} + y_{h,\omega}^{\uparrow} - z_{h,\omega}^{\uparrow} = 0, \    \forall{h>1,\omega},                                                                                                         \label{aux:1}                                       \\
        \                                                                                             & y_{h,\omega}^{\uparrow} + z_{h,\omega}^{\uparrow} \leq 1 \                                                             \forall{h,\omega}                                                                                                                                                                     \label{aux:2}                                              \\
        \                                                                                             & u_{h-1,\omega}^{\downarrow} - u_{h,\omega}^{\downarrow} + y_{h,\omega}^{\downarrow} - z_{h,\omega}^{\downarrow} = 0, \                                                                                                                                                                                                                                                    \forall{h>1, \omega},                                                                                                                                        \label{aux:3}                                       \\
        \                                                                                             & y_{h,\omega}^{\downarrow} + z_{h,\omega}^{\downarrow} \leq 1 \                                                         \forall{h,\omega}                                                                                                                                                                     \label{aux:4}                                              \\
        \                                                                                             & u_{h,\omega}^{\uparrow} + u_{h,\omega}^{\downarrow} \leq 1 \                                                           \forall{h,\omega}                                                                                                                                                                     \label{aux:5}                                              \\
        \                                                                                             & y_{h,\omega}^{\uparrow} + y_{h,\omega}^{\downarrow} \leq 1 \                                                           \forall{h,\omega}                                                                                                                                                                     \label{aux:6}                                              \\
        \                                                                                             & z_{h,\omega}^{\uparrow} + z_{h,\omega}^{\downarrow} \leq 1 \                                                           \forall{h,\omega} \label{aux:7}                                                                                                                                                                                                                  \\
        \                                                                                             & T^{\rm{f}}_{J,\omega} \leq T^{\rm{f}, \rm{Base}}_{J}, \ \forall{\omega} \label{temp:1}                                                                                                                                                                                                                                                                                     \\
        \                                                                                             & y^{\downarrow}_{h, \omega} \geq z^{\uparrow}_{h, \omega}, \                                                                                                                                                                                                                                                                 \forall{h, \omega} \label{rebound:1}        \\
        % \                                                                                             & \ y^{\downarrow}_{h, \omega} \leq z^{\uparrow}_{h, \omega}, \                                                                                                                                                                                                                                                                 \forall{h, \omega} \label{rebound:2}        \\
        \                                                                                             & \sum_{t=4(h-1)}^{4 h} T^{\rm{f}}_{t, \omega} - T^{\rm{f}, \rm{Base}}_{t} \geq \big( z^{\downarrow}_{h, \omega} -1\big)  M,  \  \forall{h>1,\omega} \label{rebound:3}                                                                                                                                                                                 \\
        \                                                                                             & \sum_{t=4(h-1)}^{4 h} T^{\rm{f}}_{t, \omega} - T^{\rm{f}, \rm{Base}}_{t} \leq \big(1 - z^{\downarrow}_{h, \omega}\big) M,  \  \forall{h>1,\omega} \label{rebound:4}                                                                                                                                                                                 \\
        \                                                                                             & \sum_{k=1}^{h} y^{\downarrow}_{k,\omega} \leq y^{\uparrow}_{k, \omega}, \ \forall{h, \omega}. \label{up_reg_first}  
    \end{align}
\end{subequations}

The objective function (\ref{P2:1}) maximizes the expected flexibility value of the freezer.
%
Constraint \eqref{P2:2} contains
(\ref{eq:bid_constraint_relaxed}), representing the McCormick relaxation of activation conditions. Recall that $\lambda^{\rm{bid}}_{h, \omega}$ in (\ref{eq:bid_constraint_relaxed}) should be replaced as defined in \eqref{eq:affine_policy}. Constraint \eqref{P2:3} declares continuous and binary variables. 
%There is a set of identical constraints to (\ref{P2:2}) that simulate the baseline temperatures, $T^{\rm{f},\rm{B}}_{t}$ and $T^{\rm{c},\rm{B}}_{t}$, using the baseline power, $P^{\rm{Base}}_{h}$.. Note, the freezer specific variables are indexed by $t$, representing a time step $dt = 0.25$ whereas all other variables are indexed by hour $h$.
%from mFRR while reducing the rebound and penalty cost for all equiprobable scenarios.
%
%Equation (\ref{P2:2}) is 
%
%Equations (\ref{P2:3}-\ref{P2:5}) are the bidding policy.
%
%Constraints \eqref{power:6}-\eqref{power:15} enforce the power consumption constraints. 
%
Aligned with (\ref{eq:2ndFreezerStateSpace}), constraints \eqref{P2:state-space-1}-\eqref{P2:state-space-2} are the state-space model for the food and air temperature dynamics. Similarly, \eqref{P2:state-space-3}-\eqref{P2:state-space-4} include the baseline air temperature $T^{\rm{c},\text{Base}}_{t}$ and the   baseline food temperature $T^{\rm{f},\text{Base}}_{t}$, and model temperature dynamics for the baseline power. %These variables are used in subsequent constraints to determine when the rebound stops. 
%Note that the freezer-specific variables are indexed by $t$, representing a time step $dt = 0.25$, whereas all other variables are indexed by hour $h$. 
Recall in case the hour index $h$ runs from 1 to 24, index $t$ runs from 1 to $J=96$.  
%
Constraint \eqref{power:6} sets the real-time power consumption $p_{h,\omega}$ equal to the baseline power $P^{\rm{Base}}_{h}$ unless there is up-regulation $p^{\rm{b}, \uparrow}_{h,\omega}$ or down-regulation $p^{\rm{b}, \downarrow}_{h,\omega}$. 
%
Constraint (\ref{power:7}) binds the mFRR reservation $p_{h}^{\rm{r},\uparrow}$ to the baseline power. 
%
Constraint (\ref{power:8}) ensures that up-regulation is zero when there is no need for up-regulation, and at the same time binds it to the reservation power. 
%
Constraint (\ref{power:9}) includes the binary variable $u^{\uparrow}_{h,\omega}$, indicating whether the freezer is up-regulated in hour $h$ under scenario $\omega$. This constraint ensures that up-regulation is zero whenever $u^{\uparrow}_{h,\omega} = 0$, and otherwise restricted to the maximum up-regulation service $P^{\rm{Base}}_{h}-P^{\rm{Min}}$ that can be provided. Note that $P^{\rm{Min}}$ is the minimum consumption level of the freezer. 
%
Constraint (\ref{power:10}) works similarly for down-regulation. Note that the binary variable $u^{\downarrow}_{h,\omega}$ indicates whether down-regulation happens, whereas $P^{\rm{Nom}}$ is the nominal (maximum) consumption level of the freezer. 
%
Constraint (\ref{power:11}) restricts the power consumption to lie within the minimum and nominal rates.
%
Constraint (\ref{power:12}) binds the slack variable $s_{h,\omega}$, representing the service not delivered as promised. 
%
Constraint (\ref{power:14}) ensures that down-regulation is equal to at least 10\% of the down-regulation capacity. 
%
Constraint (\ref{power:15}) prohibits any up-regulation when defrosting occurs.
%
Constraints \eqref{aux:1}-\eqref{aux:7} define auxiliary binary variables $y^{\uparrow}_{h,\omega}$, $y^{\downarrow}_{h,\omega}$, $z^{\uparrow}_{h,\omega}$, and $z^{\downarrow}_{h,\omega}$, identifying transitions from/to up-regulation and down-regulation.
During all hours with up-regulation, $y^{\uparrow}_{h,\omega}=1$. In the hour that up-regulation is stopped, $z^{\uparrow}_{h,\omega}$ is 1. There is a similar definition for $y^{\downarrow}_{h,\omega}$ and $z^{\downarrow}_{h,\omega}$ related to down-regulation. 
See Chapter 5 of  \cite{morales2013integrating} for complete details.
%
Constraint \eqref{temp:1} restricts the food temperature for the last time period $J$.  
%For every time period $t$, \eqref{temp:2} and \eqref{temp:3} bind the air temperature by enforcing the maximum deviation $\Delta^{\rm{max}}$.
%
Constraints \eqref{rebound:1}-\eqref{rebound:4} control the rebound behavior such that the rebound finishes when the temperature is below the baseline temperature. Note that $M$ is a sufficiently big positive constant such that the food temperature is allowed to deviate from the baseline. Also, they ensure that the rebound happens right after up-regulation.
%
Lastly, (\ref{up_reg_first}) ensures that up-regulation happens first. This makes sense since it impossible (or at least difficult) to anticipate potential up-regulation events in the power system. As such, it does not make sense to pre-cool (or pre-heat) a TCL in the context of mFRR.

% \subsubsection{Auxillary variables and constraints}\label{sec:aux_constraints}

% First, we describe the necessary auxillary variables and constraints to identify when up- and down-regulation occurs compared to the baseline power, $P^{\rm{Base}}_{h}$. This is required since the costs and revenues from up-regulation and rebound must be determined explicitly. We therefore introduce the following six binary variables \cite{morales2013integrating}:
% \\
% \begin{itemize}
%     \item $u^{\uparrow}_{h,\omega} \in \{0,1\}$ equal to 1 when starting to deliver up-regulation
%     \item $y^{\uparrow}_{h,\omega} \in \{0,1\}$ equal to 1 during up-regulation
%     \item $z^{\uparrow}_{h,\omega} \in \{0,1\}$ equal to 1 when to stopping up-regulation
%     \item $u^{\downarrow}_{h,\omega} \in \{0,1\}$ equal to 1 when starting to deliver down-regulation
%     \item $y^{\downarrow}_{h,\omega} \in \{0,1\}$ equal to 1 during down-regulation
%     \item $z^{\downarrow}_{h,\omega} \in \{0,1\}$ equal to 1 when to stopping down-regulation
% \end{itemize}

% \noindent The following constraints implements the logic:

% \begin{subequations}\label{eq:auxillary_constraints}
%     \begin{align}
%         u_{h-1,\omega}^{\uparrow} - u_{h,\omega}^{\uparrow} + y_{h,\omega}^{\uparrow} - z_{h,\omega}^{\uparrow} = 0 \         & \forall{\omega}, \forall{h} = \{2 \ldots 24 \} \\
%         y_{h,\omega}^{\uparrow} + z_{h,\omega}^{\uparrow} \leq 1 \                                                            & \forall{h,\omega}                              \\
%         u_{h-1,\omega}^{\downarrow} - u_{h,\omega}^{\downarrow} + y_{h,\omega}^{\downarrow} - z_{h,\omega}^{\downarrow} = 0 \ & \forall{\omega}, \forall{h} = \{2 \ldots 24 \} \\
%         y_{h,\omega}^{\downarrow} + z_{h,\omega}^{\downarrow} \leq 1 \                                                        & \forall{h,\omega}                              \\
%         u_{h,\omega}^{\uparrow} + u_{h,\omega}^{\downarrow} \leq 1 \                                                          & \forall{h,\omega}                              \\
%         y_{h,\omega}^{\uparrow} + y_{h,\omega}^{\downarrow} \leq 1 \                                                          & \forall{h,\omega}                              \\
%         z_{h,\omega}^{\uparrow} + z_{h,\omega}^{\downarrow} \leq 1 \                                                          & \forall{h,\omega}
%     \end{align}
% \end{subequations}

% \subsubsection{Power constraints}\label{sec:power_constraints}

% The power consumption of the freezer is constrained by:

% \begin{subequations}\label{eq:power_constraints}
%     \begin{align}
%         p_{h,\omega} = P^{\rm{Base}}_{h} - p^{\rm{b}, \uparrow}_{h,\omega} + p^{\rm{b}, \downarrow}_{h,\omega}, \                                                                                                 & \forall{h,\omega}                                                                             \label{con_power:subeq1}  \\
%         p^{\rm{r}, \uparrow}_h \leq P^{\rm{Base}}_h, \                                                                                                                                                       & \forall{h}                                                                                     \label{con_power:subeq2} \\
%         p^{\rm{b}, \uparrow}_{h,\omega} \leq p^{\rm{r}, \uparrow}_h \mathbbm{1}_{h,\omega}^{\lambda^{\rm{b}}_{h,\omega} > \lambda^{s}_{h}} , \                                                                           & \forall{h,\omega}                                                                             \label{con_power:subeq3}  \\
%         p^{\rm{b}, \uparrow}_{h,\omega} \leq u_{h,\omega}^{\uparrow} (P^{\rm{Base}}_{h} - P^{Min}) , \                                                                                                       & \forall{h,\omega}                                                                             \label{con_power:subeq4}  \\
%         p^{\rm{b}, \downarrow}_{h,\omega} \leq u^{\downarrow}_{h,\omega} (P^{\rm{Nom}} -P^{\rm{Base}}_{h}), \                                                                                              & \forall{h,\omega}                                                                             \label{con_power:subeq5}  \\
%         P^{\rm{Min}} \leq p_{h,\omega} \leq P^{\rm{Nom}}, \                                                                                                                                           & \forall{h,\omega}                                                                             \label{con_power:subeq6}  \\
%         0 \leq s_{h,\omega} \leq P^{\rm{Base}}_{h}, \                                                                                                                                                   & \forall{h,\omega}                                                                             \label{con_power:subeq7}  \\
%         p^{\rm{b}, \uparrow}_{h,\omega} + s_{h,\omega} \geq p^{\rm{r},\uparrow}_{h} \  \mathbbm{1}_{h,\omega}^{(\lambda^{\rm{bid}}_{h} < \lambda^{\rm{b}}_{h,\omega}, \lambda^{\rm{b}}_{h,\omega} > \lambda^{s}_{h})}, \ & \forall{h,\omega} \label{con_power:subeq8}                                                                              \\
%         p^{\rm{b}, \downarrow}_{h,\omega} \geq 0.10 \  u^{\downarrow}_{h,\omega} (P^{\rm{Nom}} - P^{\rm{Base}}_{h}), \                                                                                  & \forall{h,\omega}                                                                             \label{con_power:subeq9}  \\
%         p^{\rm{r}, \uparrow}_{h} \leq P^{\rm{Base}}_{h} (1 - \mathbbm{1}_{h}^{\rm{df}}), \                                                                                                                 & \forall{h} \label{con_power:subeq10}
%     \end{align}
% \end{subequations}

% Constraint (\ref{con_power:subeq1}) sets the power equal to the baseline power unless there is up- or down-regulation. Constraint (\ref{con_power:subeq2}) bounds the reservation power to the baseline power. Constraint (\ref{con_power:subeq3}) ensures that up-regulation is zero when the system does not need it, and at the same time bounds it to the reservation power. Constraint (\ref{con_power:subeq4}) ensures that up-regulation is 0 whenever $u^{\uparrow}_{h,\omega} = 0$, and otherwise bounded to the maximum power that can be upregulated. Constraint (\ref{con_power:subeq5}) works the same way for down-regulation. Constraint (\ref{con_power:subeq6}) bounds the power to be between the minimum and nominal power. Constraint (\ref{con_power:subeq7}) bounds the slack variable which is the energy not delivered as promised. Constraint (\ref{con_power:subeq8}) is the bi-linear constraint from (\ref{eq:bid_constraint}). Constraint (\ref{con_power:subeq9}) ensures that down-regulation is equal to at least 10\% of the down-regulation capacity. Lastly, constraint (\ref{con_power:subeq10}) prohibits any up-regulation when defrosting occurs.

% \subsubsection{Physical constraints}\label{sec:temperature_constraints}

% The state-space model in (\ref{eq:2ndFreezerStateSpace}) is simply added as constraints with $p_{t,\omega}$ being the power of the freezer and an additional index for each scenario, $\omega$.

% Note, the freezer specific variables are indexed by $t$, representing a time step $dt = 0.25$ whereas all other variables are indexed by hour $h$.

% Furthermore, there is a set of identical constraints to (\ref{eq:2ndFreezerStateSpace}) that simulates the baseline temperatures, $T^{f,B}_{t}$ and $T^{c,B}_{t}$, using the baseline power, $P^{Base}_{h}$. These are used for the following boundary constraint, as well as the rebound constraints in Section \ref{sec:rebound_constraints}.

% \begin{align}\label{eq:boundary_constraint}
%     T^{f}_{96,\omega} \leq T^{f, \rm{Base}}_{96}, \ \forall{\omega}
% \end{align}

% The boundary constraint in (\ref{eq:boundary_constraint}) ensures that the optimization does not exploit the end state.

% Temperature constraints to the air temperature can easily be added to limit the flexibility of the TCL by introducing a maximum temperature difference to the baseline temperature, $\Delta^{\rm{max}}$:

% \begin{subequations}\label{eq:delta_max_constraints}
%     \begin{align}
%         T^{c,\rm{Base}}_{t} - \Delta \leq T^{c}_{t, \omega}, \ & \forall{t, \omega} \\
%         T^{c,\rm{Base}}_{t} - \Delta \geq T^{c}_{t, \omega}, \ & \forall{t, \omega} \\
%         \Delta \leq \Delta^{\rm{max}}
%     \end{align}
% \end{subequations}

% \subsubsection{Rebound constraints}\label{sec:rebound_constraints}

% The rebound constraints ensures that a rebound happens right after an up-regulation by down-regulating:

% \begin{subequations}\label{eq:rebound_constraints_1}
%     \begin{align}
%         y^{\downarrow}_{h, \omega} \geq z^{\uparrow}_{h, \omega}, \ & \forall{h, \omega} \\
%         y^{\downarrow}_{h, \omega} \leq z^{\uparrow}_{h, \omega}, \ & \forall{h, \omega}
%     \end{align}
% \end{subequations}


% Furthermore, the following rebound constraints ensures that the rebound will take place until the food temperature is equal to the baseline food temperature, which can be considered the setpoint food temperature:

% \begin{subequations}\label{eq:rebound_constraints_2}
%     \begin{align}
%         \sum_{t=4\ (h-1)}^{4 h} T^{f}_{t, \omega} - T^{f, \rm{Base}}_{t} \geq - (1 - z^{\downarrow}_{h, \omega}) \  M, \ & \forall{\omega}, \forall{h} \\
%         \sum_{t=4\ (h-1)}^{4 h} T^{f}_{t, \omega} - T^{f, \rm{Base}}_{t} \leq - (1 - z^{\downarrow}_{h, \omega}) \  M, \ & \forall{\omega}, \forall{h}
%     \end{align}
% \end{subequations}

% In constraints (\ref{eq:rebound_constraints_2}), $M$ is a sufficiently big number such that the food temperature is allowed to deviate from the baseline.

% Lastly, the following constraint ensures that up-regulation happens before down-regulation:

% \begin{align}\label{eq:up_regulation_first}
%     \sum_{k=0}^{h} y^{\downarrow}_{\omega, k} \leq y^{\uparrow}_{\omega, k}, \ \forall{h, \omega}
% \end{align}

% This makes sense since it is not possible (or at least difficult) to anticipate potential up-regulation events in the power grid. As such, it does not make sense to pre-cool (or pre-heat) a TCL in the context of mFRR.


\endgroup

% \section*{Appendix A}\label{appendix:A}
% % \addcontentsline{toc}{section}{Appendices}
% % \renewcommand{\thesubsection}{\Alph{subsection}}
% % \subsection{Appendix}\label{appendix:A}

% Hao et al. \cite{hao2014aggregate} describes how a TCL can be modelled as a virtual battery using a first-order thermal-electric ODE:

% \begin{align}\label{eq:hao}
%     \frac{dT(t)}{dt} = \frac{1}{C}\left( \frac{1}{R}(T^{a}(t) - T(t)) + \eta P(t) \right)
% \end{align}

% Here, $T(t)$ is the temperature, $C$ is the thermal capacitance (kWh/$^{\circ}$C), $R$ is the thermal resistance ($^{\circ}$C/kW), $\eta$ is the coefficient of performance (COP), i.e., the cooling/heating effect, $P(t)$ is the power to the TCL, and $T^{a}(t)$ is the ambient temperature outside the TCL (typically around 20 $^{\circ}$C in an indoor environment).

% Note, (\ref{eq:hao}) can readily be formulated in a deterministic, state-space model as in (\ref{eq:sde1}). The following difference equation yields the Euler approximation of (\ref{eq:hao}) which can be used in an optimization model (with the same time step $dt$):

% \begin{align}\label{eq:hao_discretized}
%     T_{t+1} = T_t + dt\  \left( \frac{1}{C}\left( \frac{1}{R}(T^{a}_t - T_t) + \eta P_t \right)  \right)
% \end{align}

% Eq. (\ref{eq:hao}) and (\ref{eq:hao_discretized}) constitutes the most simple model of a TCL one can imagine, but, nevertheless, has a quite powerful interpretation: the rate of change of temperature is determined by the heat flux to the surrounding environment and the heat flux from the power source to the TCL. It thus captures the most fundamental temperature dynamics of a TCL.

% The steady-state power in (\ref{eq:hao}) is given by:

% \begin{align}\label{eq:hao_ss}
%     P^{ss}(t) = \frac{T^{a}(t) - T(t)}{\eta R}
% \end{align}

% The steady-state power is thus the power required to keep the temperature of the TCL constant with respect to the outside temperature, $T^{a}(t)$, given the efficiency of the system and the thermal resistance. A better energy efficiency can be achieved by either 1) increasing the mechanical efficiency of the cooling/heating system or 2) increasing the thermal resistance to the outside temperature by, e.g., insulating a freezer.

% The drawback of the first-order model in (\ref{eq:hao}) is that it is only parameterized by three parameters, and it excludes disturbances. Hence, it might not be an accurate model of a real system. The model can easily be extended to include more complicated dynamics such as heat exchange with a barrier between $T^{a}(t)$ and $T(t)$, additional disturbance terms (e.g., when a fridge is opened), hourly values of $C$ and $R$, etc. Nevertheless, (\ref{eq:hao}) serves as a good starting point for a simple TCL model.

% \section*{Appendix B}\label{appendix:B}
% Appendix two text goes here.
}


\input{../sections/acknowledgement}


\bibliographystyle{IEEEtran}
\bibliography{../bibliography/Bibliography}

% \newpage

% \begin{IEEEbiographynophoto}{Peter A.V. Gade}
%     is an Industrial PhD researcher at IBM and affiliated with the Technical University of Denmark, Kongens Lyngby, Denmark, in the Energy Markets and Analytics Section within the Power and Energy Systems division at the Wind and Energy Systems Department. His research focuses on demand-side flexibility and the revenue streams from utilization of demand-side flexibility. He holds a M.S. in Mathematical Modelling and Computing and a B.S. in Biomedical Engineering, both from the Technical University of Denmark.
% \end{IEEEbiographynophoto}
% \begin{IEEEbiographynophoto}{Trygve Skjøtskfit}
%     is an Associate Partner at IBM Denmark, with focus on energy transformation and demand-side flexibility. His solid experience and deep knowledge within intelligent energy systems, buildings, and civil infrastructures makes him a leading figure, strategic advisor, and a first mover in the flexibility market with a strong track record to find and deliver new cutting-edge solutions. He holds an MBA in Strategy from Universitat Pompeu Fabra, and a Master of Export Engineering from Copenhagen University, College of Engineering.
% \end{IEEEbiographynophoto}
% \begin{IEEEbiographynophoto}{Henrik W. Bindner}
%     received the MSc in Electrical Engineering from Technical University of Denmark in 1988. He is currently a senior researcher with the Department of Wind and Energy Systems, Technical University of Denmark. He is heading the \textit{Distributed Energy Systems} Section and his research interests include control and management of smart grids, active distribution networks, and integrated energy systems.
% \end{IEEEbiographynophoto}
% \begin{IEEEbiographynophoto}{Charalampos Ziras}
%     ...
% \end{IEEEbiographynophoto}
% \begin{IEEEbiographynophoto}{Jalal Kazempour}
%     is an Associate Professor with the Department of Wind and Energy Systems, Technical University of Denmark, where he is heading the \textit{Energy Markets and Analytics} Section. He received the Ph.D. degree in Electrical Engineering from the University of Castilla-La Mancha, Ciudad Real, Spain, in 2013. His research interests include intersection of multiple fields, including power and energy systems, electricity markets, optimization, game theory, and machine learning.
% \end{IEEEbiographynophoto}


\vfill

\end{document}
