% \section*{Abstract}
This paper investigates how a  thermostatically controlled load can deliver flexibility either in form of manual frequency restoration reserves (mFRR) or load shifting, and which one is financially more appealing to such a load. %We discuss advantages and disadvantages of the two and their appeal from a monetary point of view.
A supermarket freezer is considered as a representative flexible load, and a grey-box model  describing its temperature dynamics is developed using real data from a supermarket in Denmark. Taking into account price and activation uncertainties, a two-stage stochastic mixed-integer linear program is formulated to maximize the flexibility value from the freezer. For practical reasons, we propose a linear policy to determine regulating power bids, and then linearize the mFRR activation conditions through the  McCormick relaxation approach. For computational ease, we develop a decomposition technique, splitting the problem to a set of smaller sub-problems, one per scenario. %An out-of-sample evaluation was done on unseen Danish market price data for 2022. 
%
%We observed that load shifting was more profitable, but had a greater impact on the air and food temperatures in the freezer as opposed to mFRR that depends on the system state and bid price for activation.
%
%solution strategies for mFRR are presented: one with a simple policy and the other one with a dynamically updated policy. The proposed model accurately captures the timeline of decisions for bidding in manual frequency reserves.
%by using McCormick relaxation.
Examined on an out-of-sample simulation based on real Danish spot and balancing market prices in 2022, load shifting shows to be more profitable than mFRR provision, but is also more consequential for temperature deviations in the freezer.
