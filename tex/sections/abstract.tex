% \section*{Abstract}
This paper investigates how a  \blue{thermostatically controlled} load can deliver flexibility \blue{either in form of manual frequency reserves or load shifting, and which one is financially more appealing for such a load.} %We discuss advantages and disadvantages of the two and their appeal from a monetary point of view. 
\blue{A supermarket freezer is consideblue as a representative flexible load, and a grey-box model  describing its temperature dynamics is developed} using real data from a supermarket in Denmark. \blue{Taking into account price and activation uncertainties,} a two-stage stochastic mixed-integer linear program is formulated to maximize the flexibility value from the freezer, where two solution strategies for manual frequency reserves are presented: one with a simple policy and \blue{the other one} with a dynamically updated policy. The \blue{proposed} model accurately captures the timeline of decisions for bidding in manual frequency reserves.
%by using McCormick relaxation. 
\blue{Examined on an ex-post out-of-sample simulation,} load shifting shows to be more profitable than \blue{the provision of} manual frequency reserves, but is also more consequential for the temperature in the freezer than flexibility provision for manual frequency reserves.
