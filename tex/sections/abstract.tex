% \section*{Abstract}

This paper investigates how a flexible load can deliver flexibility for manual frequency reserves and load shifting. We discuss the advantages and disadvantages of the two and their appeal from a monetary point of view. To this end, a grey-box model of a supermarket freezer is created to describe its temperature dynamics using real data from a supermarket in Denmark. A two-stage stochastic mixed-integer linear program (MILP) maximizes the flexibility value from the freezer where two solution strategies for manual frequency reserves are presented: one with a simple policy and one with a dynamically updated policy. The model accurately captures the timeline of decisions for bidding in manual frequency reserves by using McCormick relaxation. Load shifting shows to be more profitable than manual frequency reserves on unseen price data, but is also more consequential for the temperature in the freezer than flexibility provision for manual frequency reserves.
