\section{Introduction}

Power consumers are looking to reduce costs by any means due to sky-rocketing power- and gas prices. In Denmark, supermarkets are especially exposed due to their high energy consumption, and they are actively looking into initiatives to reduce their electricity bill. One of these initiatives is demand-side flexibility, whereby a supermarket can shift consumption in time when spot prices are lower or when the power grid needs up-regulation. There are thousands of supermarkets in Denmark, willing to provide flexibility, and their thermostatically controlled loads (TCLs) are especially suited for this. However, the monetary benefit still remains uncertain. In this paper, we describe how to estimate and utilize a supermarket freezer's flexibility with respect to manual frequency reserves and load shifting.

There are also a lot of initiatives in Denmark to harness demand-side flexibility. For example, IBM is currently developing a software platform - called Flex Platform - that can harness flexibility from industrial and commercial consumers for bidding in ancillary service markets by a balance responsible party (BRP). One large supermarket chain in Denmark has already signed up for the Flex Platform, and another supermarket chain is currently exploring the possibility of joining. In order to understand the flexibility of freezers, one must first understand the working of freezers, their temperature dynamics, and how they are controlled.

To this end, many studies have used grey-box models to capture the most fundamental physics while using data to calibrate the model. A grey-box model can illustrate the concept of flexibility well, and we show how it can be used for simulations. Such a model can also readily be used in a mathematical optimization framework to, e.g., reduce costs or maximize earnings from flexibility. An optimization model can be used to obtain a simple policy for how to shift consumption in time and how to provide flexibility in the form of manual frequency reserves, also called mFRR\footnote{\textbf{mFRR}: manual frequency restoration reserves. The term will be used for the remainder of the paper and is equivalent to tertiary reserves.}. In this work, we determine the monetary benefit of flexibility using such a optimization model with two solution strategies: 1) one using a five days lookback and 2) one using 2021 data. We evaluate the model on unseen 2022 price data.

\IEEEpubidadjcol


\subsection{Dilemma: mFRR vs load shifting}

Furthermore, for consumers that are actively looking to reduce costs, it is important to understand the monetary benefit of flexibility. From their perspective, it is natural to first look into load shifting at first since it is the most obvious and easy way to utilize flexibility. Here, load shifting simply refers to shifting consumption to a point in time, typically when spot prices are lower.

However, load shifting is not necessarily helping the power grid. In fact, if a significant amount of consumers started to do it, it could be detrimental to the power grid by moving peak consumption to a different time of the day. Intriguingly, many industrial and residential consumers have already started doing that to some degree as a response to the energy crisis. As a result, load shifting might provide \textit{short-term} monetary benefit, but it is not necessarily helping the power grid in the long run.

Instead, consumers could opt for ancillary services for which they are paid for providing flexibility through an \textit{aggregator}. One such service is mFRR in Denmark, which is a slow-responding reserve used to stabilize the frequency in the power grid. mFRR resembles load shifting in the sense that it is the largest energy reserve operated by the transmission system operator (TSO), and can be provided by consumers by shifting their consumption in time.\footnote{Likewise, generators provide mFRR by increasing their power generation.}

The demand response for consumers participating in mFRR versus load shifting will therefore look similar. From the consumers' perspective, it is of great interest to investigate the monetary benefit of participating in each, and it is important to understand the incentive of consumers to participate in mFRR from a TSO perspective as well.

\subsection{Research questions}

To understand the incentives to participate in load shifting and mFRR from a flexible consumer's perspective, this work aim to investigate and answer the following research questions:

\begin{itemize}
    \item How can a supermarket freezer's flexibility be characterized?
    \item How much monetary benefit can a freezer provide for mFRR?
    \item How much monetary benefit can a freezer provide for load shifting?
    \item What are the advantages and disadvantages of participating in mFRR versus load shifting?
\end{itemize}

Some of these questions have been looked at extensively in literature, so we first provide a brief literature review to pinpoint where new work is required and how our work relates to existing work.

\subsection{Literature review and our contributions}

Demand-side flexibility has been extensively studied. The approach taken in each study is very much dependent on whose perspective from which the flexibility is utilized. Often, full knowledge and control is assumed. An often overlooked, implicit assumption is that the aggregator and/or trading entity are the same with an exclusive business relationship to flexible consumer \cite{gade2022ecosystem}. The incentives and investments required for delivering flexibility for the flexible consumer are often overlooked as well. Many also assume an idealized market mechanism or simply propose a new mechanism for trading flexibility while we look at the status quo today.

In \cite{shafiei2013modeling}, a complete white-box model of a supermarket refrigeration system is presented and validated against real-life data. It is also shown how such a system can provide a demand-response. This work serves as a good benchmark for any grey-box model of a supermarket refrigeration system. However, such an approach is hard to scale and requires complete knowledge of every refrigeration system. A very similar approach is taken in \cite{petersen2012eso2} and \cite{pedersen2013direct}. In \cite{pedersen2016improving}, a second-order model is used to model the food and air temperature in a freezer in an experimental setting. It is shown how the food temperature has much slower dynamics than the air temperature. In \cite{hao2014aggregate}, a simple first-order virtual battery model for a TCL is presented. This model also constitutes the starting point for modelling temperature dynamics in this report as it has a powerful, physical interpretation. A similar \textit{bucket} model is introduced in \cite{petersen2013taxonomy}.

In \cite{sossan2016grey}, the grey-box modelling approach to a refrigeration system is described in detail from formulation to validation. They use the grey-box model for model predictive control (MPC) for load shifting. In \cite{o2013modelling}, an ARMAX time-series model is used to characterize the temperature development of supermarket freezers and refrigerators. It is validated on one-step prediction errors, and the authors also note how it is able to roughly simulate the system over a longer period as well. This classical time-series approach provides an alternative to the state-space approach. The authors show how MPC with three different objectives can be used to optimize the flexibility in the system using the ARMAX model.

In \cite{de2019leveraging}, a MILP formulation is used to specifically address up- and down-regulation hours from a baseline consumption. It is assumed that the energy for down-regulation is equal to the energy not consumed when up-regulating. A similar approach is taken in our work except we will define use down-regulation until the temperature state is (approximately) back to its setpoint.


In \cite{schaperow2019simulation}, \cite{chanpiwat2020using}, \cite{moglen2020optimal}, and \cite{moglen2020optimal}, residential air condition units are modelled using up- and down-regulation blocks characterizing the flexibility. The blocks are obtained from grey-box models of households \cite{siemann2013performance}. The authors show how such a block formulation can be solved using exact and stochastic dynamic programming in the context of peak shaving demand for a utility in the US. The demand-side flexibility then functions as a hedge towards extreme electricity prices. Although such a use case has not been relevant in Denmark yet, the estimation of flexibility blocks is a novel idea as it avoids having to explicitly integrate a physical model into an optimization problem. However, the tradeoff is that the curse of dimensionality quickly makes dynamic programming computationally intractable for a large portfolio of heterogenous demand-side assets. Furthermore, it might be difficult to assure the Markov property \cite{MarkovProperty} when reformulating the problem even slightly.

Flexibility blocks are also used in \cite{bobo2018offering} for an offering strategy. Here, the flexibility blocks were derived from measurements of residential appliances in the Ecogrid 2.0 project \cite{ecogrid}.

In \cite{biegel2013information} and \cite{BiegelConstractingFlexServices}, the authors describe how flexibility interfaced with pre-defined flexibility characteristics allow aggregators to utilize and contract the flexibility. This assumes flexibility of the demand-side assets is fully known beforehand. However, this type of information flexibility interface and flexibility contract specification is still useful for aggregators when engaging with consumers with well-known assets at the point where aggregation of demand-side flexibility is a mature and prevalent business.

In \cite{biegel2013electricity} a portfolio of residential heat pumps are modelled using a linear, first-order model. Individual heat pumps are lumped together as essentially one big, aggregated heat pump as described in \cite{biegel2013lumped}. The first-order model is a grey-box model incorporating the physics, i.e., the temperature dynamics in household. Such an approach will be used in this work as well. The authors also consider how the portfolio of heat pumps can be used for spot price optimization and real-time bidding in the balancing market but do not consider mFRR. However, they assume that all the assets can be controlled continuously in an economic model predictive control (E-MPC) setting. The temperature deviations are minimized directly in the objective function using the integrated error of total temperature deviations. This introduces a trade-off parameter in the objective function that must be tuned to weigh economic value versus temperature deviation to the setpoint. A close, integrated relationship between the BRP, aggregator, and consumer is assumed.

To the best of our knowledge, there is no work that investigated the monetary incentives for providing mFRR versus load shifting, respectively, although both have been studied extensively. This work aims to fill that gap. Furthermore, when stating the objective functions for mFRR, most studies have assumed a simplified market structure, and neglect the fact that the aggregator and the consumer are not necessarily the same entity. We discuss the consequences of this assumption in relation to the market structure in Denmark \cite{gade2022ecosystem}, and in the context of mFRR and load shifting.

\subsection{Outline}

The rest of the paper is organized as follows. In Section \ref{sec:monetizing_flex}, modelling of TCLs using grey-box models is described using real data of a supermarket freezer in Denmark. Thereafter, mFRR and load shifting are described in detail and their objective functions are stated. In Section \ref{sec:OptimizationModel}, the optimization problem is presented, both in compact and final form. Furthermore, two solution strategies for solving the optimization problem for mFRR are introduced. In Section \ref{sec:results}, the results of the case study are presented and discussed. Finally, Section \ref{sec:conclusion} concludes the paper.
