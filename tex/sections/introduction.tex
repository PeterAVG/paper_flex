\vspace{-0.2cm}
\section{Introduction}

Power consumers are looking to reduce costs by any means due to sky-rocketing energy and gas prices. In Denmark, supermarkets are especially exposed due to their high energy consumption, and they are actively looking into initiatives to reduce their electricity bill. One of these initiatives is the application of demand-side flexibility, whereby a supermarket can shift consumption in time when spot (day-ahead) prices are comparatively lower or when the power system needs up-regulation services. There are thousands of supermarkets in Denmark willing to provide flexibility, and their freezers are especially suited for this as they constitute a big share of their energy consumption.

IBM is currently developing a software platform, called \textit{Flex Platform}, aiming to harness flexibility from industrial and commercial consumers for bidding in ancillary service markets via a Balance Responsible Party (BRP). One large supermarket chain in Denmark has already signed up for the \textit{Flex Platform}, and another supermarket chain is currently exploring the possibility of joining.
While these supermarket chains are keen on the provision of flexibility, the estimation of their monetary benefit still remains an open question. We aim to address it by considering two potential ways for supermarkets to exploit their consumption flexibility: (\textit{i}) load shifting, and (\textit{ii}) the provision of manual Frequency Restoration Reserve (mFRR)
%\footnote{The term will be used for the remainder of the paper and is equivalent to tertiary reserves.})
services. We  develop  a stochastic optimization tool capturing the temperature dynamics and therefore consumption flexibility of supermarkets, and compute the monetary benefit of flexibility implemented via load shifting or mFRR service provision. We use two scenario generation  strategies: one using a five-day lookback on spot and balancing market prices in Denmark, and the other one using those price data in 2021. We then systematically compare these two strategies ex-post via an out-of-sample simulation based on price data in 2022.

%In this paper, we describe how to utilize a supermarket freezer's flexibility with respect to manual frequency restoration reserves (mFRR\footnote{The term will be used for the remainder of the paper and is equivalent to tertiary reserves.}) and load shifting, and thereby the incentives for doing so.

%Many studies have used grey-box models to capture the most fundamental physics while using data to calibrate the model. A grey-box model can illustrate the concept of flexibility well, and we show how it can be used for simulations. Such a model can also readily be used in a mathematical optimization framework to, e.g., reduce costs or maximize earnings from flexibility. An optimization model can be used to obtain a simple policy for how to shift consumption in time and how to provide flexibility for mFRR.



%Our contribution is two-fold. \blue{First,} we show how to model mFRR reservation, bidding, and activation using a novel two-stage stochastic program and 2) we investigate and discuss the incentives to utilize demand-side flexibility for mFRR versus load shifting which is interesting from both a power system- and consumer perspective. % For example, if the consumers are incentivized to load shift instead of providing mFRR, the power system will not necessarily benefit, consumers will only benefit in the short-term.

\IEEEpubidadjcol

\vspace{-1mm}
\subsection{Dilemma: Load Shifting Versus mFRR}
%
For power consumers who are actively looking into cost saving solutions, it is important to estimate the monetary benefit of flexibility. In Nord Pool, spot prices are announced at 2pm for the next day, so supermarkets who purchase power through a retailer at  spot prices, have a chance to shift their consumption plan for the next day, accordingly. From their perspective, it is natural to first look into load shifting as the most obvious and straightforward way to utilize flexibility. Here, load shifting simply refers to shifting consumption to another point in time, typically when spot prices are lower.

Nevertheless, load shifting, although might be attractive individually, is not necessarily in  favor of the power system.
If a significant number of consumers start to act similarly, it could be detrimental to the power system by moving the peak consumption to a different time of the day. In this case, spot prices are no longer reflecting supply-demand equilibrium, and the demand distribution over the day.
%In fact, if a significant amount of consumers move a large part of their consumption to a different time of the day, that could be detrimental to the power system by creating new and even more pronounced peaks.
Intriguingly, many industrial and residential consumers have already started shifting demand to some degree, as a response to the energy crisis.
As a result, load shifting might provide short-term monetary benefit, but it is not necessarily helping the power system in the long run.

%Instead, consumers could opt for offering ancillary services, for which they are paid for providing flexibility through an \textit{aggregator}.
As an alternative option which is certainly in favor of the power system, consumers could choose to provide flexibility by offering ancillary services through an \textit{aggregator}, for which they would be compensated. Here, we focus on frequency-supporting services, and in particular mFRR, and left a potential extension by considering more services for the future work. This service is a slow-responding reserve used to stabilize frequency in the power system after fast frequency reserves are depleted. The market for mFRR is usually operated by the national Transmission System Operator (TSO), which is Energinet in Denmark. Note that mFRR resembles load shifting in the sense that it is the largest energy reserve operated by the TSO, and can be provided by consumers via shifting their consumption in time. Likewise, generators provide mFRR by increasing their power generation.

In order for a supermarket freezer to deliver mFRR, it must be part of a larger portfolio with potentially many small assets to meet the minimum bid size requirement for entering the mFRR market. In Denmark, this minimum bid size is currently 5 MW, but it is expected to reduce to 1 MW in 2024. Furthermore, there might be additional synergy effects such as increased operational flexibility in the control response, rebound, and temperature deviations \cite{koch2011modeling}. 

For simplicity, we assume that one freezer delivering mFRR can be scaled to many freezers in the same way. Therefore, 
%(and likewise for load shifting).
demand response for consumers participating in the mFRR market versus load shifting will  look similar. From the perspective of consumers, it is of great interest to explore the monetary benefit of mFRR versus load shifting. Similarly, it is important for TSOs to understand the incentive of consumers to provide mFRR services opposed to simply shifting load.

\vspace{-1mm}
\subsection{Research questions and literature review}
%
This work aims to investigate and answer the following research questions: (\textit{i}) how can the flexibility of a supermarket freezer be characterized? (\textit{ii}) how much monetary benefit can be earned by the mFRR provision opposed to load shifting? and finally (\textit{iii})
what are the pros and cons of participating in the mFRR market versus load shifting? Some of these questions have been  looked at  in the literature, so we provide a  literature review to pinpoint where new work is required, and how our work differs from the current literature.



%Demand-side flexibility has been extensively studied.
The approach taken in each study in the literature is very much dependent on the perspective from which flexibility is utilized. Often, full knowledge and control of all assets is assumed. A prevalent implicit assumption is that the aggregator and the trading entity are the same, with an exclusive business relationship to flexible consumers \cite{gade2022ecosystem}. The incentives and investments required for delivering flexibility for the flexible consumer are often overlooked as well. Many also assume an idealized market mechanism or simply propose a new mechanism for trading flexibility.

In \cite{petersen2012eso2} and \cite{pedersen2013direct}, a complete white-box model of a supermarket refrigeration system is presented and validated against real-life data. It is also shown how such a system can provide demand response. This work serves as a  benchmark for any grey-box model of a supermarket refrigeration system. However, such an approach is hard to scale and requires complete knowledge of every refrigeration system. In \cite{pedersen2016improving}, a second-order model is used to model the food and air temperature in a freezer in an experimental setting. It is shown how food temperature has much slower dynamics than air temperature. In \cite{hao2014aggregate}, a simple first-order virtual battery model for a Thermostatically Controlled Load (TCL) is presented. This model also constitutes the starting point for modeling temperature dynamics in this work, as it has an intuitive interpretation. A similar bucket model is introduced in \cite{petersen2013taxonomy}.

%In \cite{sossan2016grey}, the grey-box modeling approach to a refrigeration system is described in detail from formulation to validation. They use the grey-box model for model pblueictive control (MPC) for load shifting. In \cite{o2013modeling}, an ARMAX time-series model is used to characterize the temperature development of supermarket freezers and refrigerators. It is validated on one-step pblueiction errors, and the authors also note how it is able to roughly simulate the system over a longer period as well. This classical time-series approach provides an alternative to the state-space approach. The authors show how MPC with three different objectives can be used to optimize the flexibility in the system using the ARMAX model.

In \cite{de2019leveraging}, a Mixed-Integer Linear Programming (MILP) problem  is developed to  address up- and down-regulation hours from a baseline consumption. It is assumed that the energy for down-regulation is equal to the energy not consumed when up-regulating. A similar approach is taken in our work, except we will define the use of down-regulation until the temperature state is (approximately) back to its setpoint.

In \cite{schaperow2019simulation}, \cite{chanpiwat2020using}, \cite{moglen2020optimal}, and \cite{moglen2020optimal}, residential air condition units are modeled using up- and down-regulation blocks characterizing flexibility. The blocks are obtained from grey-box models of households \cite{siemann2013performance}. The authors show how such a block formulation can be solved using exact and stochastic dynamic programming in the context of peak shaving demand for a utility in the US. The demand-side flexibility then functions as a hedge towards extreme electricity prices. Although such a use case has not been relevant in Denmark yet, the estimation of flexibility blocks is a novel idea as it avoids having to explicitly integrate a physical model into an optimization problem. However, a tradeoff is that the curse of dimensionality quickly makes dynamic programming computationally intractable for a large portfolio of heterogeneous demand-side assets. Furthermore, it might be difficult to assure the Markov property \cite{MarkovProperty} when reformulating the problem even slightly. Flexibility blocks are also used in \cite{bobo2018offering} for an offering strategy. Here, the flexibility blocks are derived from measurements of residential appliances in the Ecogrid 2.0 project \cite{ecogrid}.

In \cite{biegel2013information} and \cite{BiegelConstractingFlexServices}, it is described how pre-defined asset characteristics allow aggregators to utilize and contract their flexibility through an interface. This assumes that such characteristics are fully known beforehand. However, this type of information flexibility interface is still useful for aggregators when engaging with consumers with well-known assets at the point where aggregation and penetration of demand-side flexibility is a mature and prevalent business.

In \cite{biegel2013electricity}, a portfolio of residential heat pumps is modeled using a linear first-order model. Individual units are lumped together as essentially one big, aggregated heat pump as described in \cite{biegel2013lumped}. The first-order model is a grey-box model incorporating the physics, i.e., the temperature dynamics in a household. Such an approach will be used in this work as well. The authors also consider how the portfolio of heat pumps can be used for load shifting and real-time bidding in the balancing market, but do not consider mFRR service provision. However, they assume that all the assets can be controlled continuously in a  model predictive control  setting. Temperature deviations are minimized directly in the objective function using the integrated error of total deviations. This introduces a trade-off parameter in the objective function that must be tuned to weigh the economic value versus the temperature deviation compared to the setpoint.

\vspace{-1mm}
\subsection{Our contribution and outline}
%
To the best of our knowledge, no work has investigated the monetary value and incentives for providing mFRR services versus load shifting, although both have been studied extensively. This work aims to fill that gap by taking a holistic view on the incentives from the flexible consumer's perspective. Furthermore, when stating the objective functions for mFRR, most studies have assumed a simplified market structure, and neglect the fact that the aggregator and the consumer are not necessarily the same entity. We discuss the consequences of this assumption in relation to the market structure in Denmark \cite{gade2022ecosystem}, and in the context of mFRR and load shifting.

We provide a realistic model formulation of the mFRR bidding, taking into account the sequence of decisions needed to be made. We propose two solution strategies for mFRR bidding: (\textit{i}) A simple one learned using the Alternating Direction Method of Multipliers (ADMM) on 2021 price data which can be applied for all of 2022, and (\textit{ii}) A dynamic one which computes a new policy every day by looking at spot prices for the past five days. Real data of a supermarket freezer  in Denmark is used.


The rest of the paper is organized as follows. Section \ref{sec:monetizing_flex} describes modeling a supermarket freezer as a TCL. Thereafter, mFRR and load shifting are described in detail and their objective functions are stated. Section \ref{sec:OptimizationModel} presents the proposed stochastic optimization problem. 
%Furthermore, the two solution strategies for solving the optimization problem for mFRR are introduced. 
Section \ref{sec:results} provides  results for a case study. Section \ref{sec:conclusion} concludes the paper. Finally, an Appendix provides the full model formulation. 
