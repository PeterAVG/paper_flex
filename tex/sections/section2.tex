\section{Monetizing flexibility from TCLs}\label{sec:monetizing_flex}

There are several ways to monetize flexibility from TCLs. In this section, we focus on mFRR and load shifting. First, we describe how to mathematically model a TCL as a flexible resource. Second, we describe how to monetize the flexibility from TCLs for mFRR and load shifting and provide the objective functions in both cases. For mFRR, the objective function include all costs and revenues for the BRP, while for load shifting, the objective function only includes the flexible demand's perspective. This approach explicitly shows the situation where flexible demand take actions without including the BRP as this is more realistic.

\subsection{Modelling TCL as a flexible resource}

TCLs are characterized by being controlled such that the temperature is kept at a specified setpoint. Examples includes heat pumps, freezers, air condition units, ovens, etc. They are widely believed to constitute an important part of demand-side flexibility due to the inherent thermal inertia of such temperature-driven systems.

In this paper, we focus on freezers, which are a very common type of TCLs. Specifically, we focus on a single freezer display in a Danish supermarket. Freezers are characterized by a large thermal inertia due to the frozen food, which makes them suitable for flexibility. On the other hand, there is a risk of food degradation when utilizing flexibility. Therefore, it is important to model the temperature dynamics in the freezer for a realistic and risk-aware estimation of its flexibility.

The rest of the section is organized as follows. First, we visualize the measurements from a real supermarket freezer. Second, we introduce a second-order grey-box model that characterizes the supermarket freezer. Third, we validate the second-order model and show how it can be used to simulate a demand response from a freezer.

\subsubsection{Supermarket freezer}

In this paper, data from a large Danish supermarket for a single freezer is used as a case study. In Figure \ref{fig:chunk}, the temperature, opening degree, and power of the freezer is shown as an average for all hours in a day over a year. The temperature fluctuates around its setpoint at -18 $^{\circ}$C with the exception of hour 7-8 where defrosting is scheduled. While defrosting, a heating element is turned on briefly, and the expansion valve is closed such that the flow of refrigerant stops. Afterwards, while recovering the temperature, the expansion valve is opened fully again. The power consumption of the whole compressor rack is shown in the bottom plot. The power consumption is highest during opening hours, and it is lowest during closing hours. During opening hours, food is being replaced and customers open the display case constantly. Furthermore, most supermarkets puts additional insulation on the display cases during closing hours which reduced thermal losses. For these reasons, there are effectively two regimes for a supermarket freezer plus a short defrosting regime.

\begin{figure}[!t]
    \centering
    \includegraphics[width=\columnwidth]{../figures/tmp_od_Pt.png}
    \caption{\textbf{Top}: average temperature of a single freezer in a supermarket. \textbf{Middle}: average opening degree of the freezer expansion valve. \textbf{Bottom}: average power of the compressors feeding all the freezers (scaled to only include all freezers in the supermarket).}
    \label{fig:chunk}
\end{figure}

\subsubsection{Thermal modelling of freezer}

In \cite{hao2014aggregate}, it is described how a simple TCL model can be made. We extend it to a second-order model that accounts for the thermal mass of the food, which essentially provides the flexibility in freezers:

\begin{subequations}\label{eq:2ndFreezer}
    \begin{align}
        \frac{dT^f(t)}{dt} & = \frac{1}{C^f}\left(\frac{1}{R^{cf}} (T^c(t) - T^f(t)) \right) \\
        \frac{dT^c(t)}{dt} & = \frac{1}{C^c}\Bigl(\frac{1}{R^{cf}} (T^f(t) - T^c(t)) \notag  \\ & + \frac{1}{R^{ci}(t)} (T^i(t) - T^c(t))                                        \notag  \\ &  -  \frac{1}{n} \eta \cdot OD(t) P(t) \Bigr) + \epsilon \mathbbm{1}^{df}
    \end{align}
\end{subequations}

In state-space form, the system in (\ref{eq:2ndFreezer}) is:

\begin{subequations}\label{eq:2ndFreezerStateSpace}
    \begin{align}
        T^{f}_{t+1} & = T^{f}_{t} + dt \cdot \frac{1}{C^f}\left(\frac{1}{R^{cf}} (T^{c}_{t} - T^{f}_{t}) \right)                                                                              \\
        T^{c}_{t+1} & = T^{c}_t + dt \cdot \frac{1}{C^c}\Bigl(\frac{1}{R^{cf}} (T^{f}_t - T^{c}_t) + \frac{1}{R^{ci}_{t}} (T^{i}_t - T^{c}_t)                                          \notag \\ & \mspace{50mu} - \frac{1}{n} \eta \cdot OD_t P_t \Bigr) + \epsilon \mathbbm{1}^{df}
    \end{align}
\end{subequations}


Here, $T^c$ is the air temperature in the freezer, and $T^f$ is the food temperature which is a latent, unobserved state. It is essentially a low-pass filter of the air temperature in the freezer with time constant $\tau = C^f R^{cf}$. $C^f$ and $C^c$ are the thermal capacitance of the food and air in the freezer, respectively. $R^{cf}$ and $R^{ci}$ are the thermal resistance between food and air in the freezer, and air and indoor temperature, respectively. Furthermore, $\epsilon$ represents the temperature change when defrosting and $\mathbbm{1}^{df}$ is an indicator for when defrosting happens. $R^{ci}$ is time-varying to capture the differences between opening- and closing hours. The opening degree, $OD_t$, and power $P_t$, are exogenous inputs. In this work, only $P_t$ is controllable. $n$ is the number of freezers in the supermarket, and $\eta$ is the compressor efficiency. The model is discretized with a time step of 15 minutes, i.e. $dt = 0.25$ hours.

\subsubsection{Model validation}

Using the R library CTSM-R \cite{juhl2016ctsmr}, the parameters in (\ref{eq:2ndFreezerStateSpace}) have been estimated as shown in Table \ref{tab:parameter_estimates}. Notice that the thermal capacitance of the air in the freezer is significantly smaller than the thermal capacitance of the food, indicating that that the food temperature changes comparatively slower. The thermal resistance between the food and air inside the freezer, $R^{cf}$, is also significantly smaller than the thermal resistance between the air in the freezer and the indoor temperature in the supermarket, $R^{ci}$, both during the day and the night. This makes sense as the lid acts as a physical barrier insulating the freezer. Furthermore, the thermal resistance to the indoor air temperature is higher during the night which means that less power is needed as seen in Figure \ref{fig:chunk}.

\begin{table}[!t]
    \caption{Parameter Estimates of (\ref{eq:2ndFreezerStateSpace}).}
    \label{tab:parameter_estimates}
    \centering
    \begin{tabular}[b]{|l|l|l|}
        \hline
        Parameter       & Value & Unit            \\ \hhline{|=|=|=|}
        $C^f$           & 5.50  & kWh/$^{\circ}$C \\
        $C^c$           & 0.13  & kWh/$^{\circ}$C \\
        $R^{cf}$        & 4.91  & $^{\circ}$C/kW  \\
        $R^{ci, day}$   & 25.6  & $^{\circ}$C/kW  \\
        $R^{ci, night}$ & 46.5  & $^{\circ}$C/kW  \\
        $\eta$          & 2.38  &                 \\
        $\epsilon$      & 6.477 & $^{\circ}$C/h   \\ \hline
    \end{tabular}
\end{table}


The model residuals calculated from the one-step ahead prediction errors should ideally resemble white noise in order for a model to capture all dynamics \cite{madsen2007time}. Figure \ref{fig:2ndFreezerModelValidation} shows the auto-correlation and cumulated periodogram of the residuals. The autocorrelation shows two significant lags for lag two and seven, but looks good otherwise. Likewise, in the periodogram, it seems the model is able to capture most dynamics at all frequencies.

\begin{figure}[!t]
    \centering
    \includegraphics[width=\columnwidth]{../figures/2ndFreezerModelValidation.png}
    \caption{ Validation of the state-space model in (\ref{eq:2ndFreezerStateSpace}). \textbf{Left}: auto-correlation function of the model residuals. \textbf{Right}: cumulated periodogram of the residuals.}
    \label{fig:2ndFreezerModelValidation}
\end{figure}

Furthermore, Figure \ref{fig:2ndFreezerModelSimulation} (left) shows a 24-hour simulation of (\ref{eq:2ndFreezerStateSpace}). It is seen that the simulation is very reasonable and closely follows the measured air temperature. However, it is quite difficult to capture the dynamics when, and immediately after, defrosting. Nevertheless, the model residuals appears to resemble white noise otherwise, and the simulation is accurate as well. It is therefore deemed that the model is good enough to proceed.

In Figure \ref{fig:2ndFreezerModelSimulation} (right), an example of a demand response event is shown. It can clearly be seen how the air temperature increases when the power is turned off, and how it decreases when the power is turned back on. The food temperature is much more stable and only changes slightly, as expected. The rebound occurs until the food temperature is back to its normal value.

\begin{figure}[!t]
    \centering
    \includegraphics[width=\columnwidth]{../figures/2ndFreezerModelSimulation.png}
    \caption{ \textbf{Left}: Simulation of (\ref{eq:2ndFreezerStateSpace}) using the parameters in Table \ref{tab:parameter_estimates}. \textbf{Right}: Simulation where power is turned off for two hours with a subsequent rebound at the nominal power until the food temperature is back to its normal value.}
    \label{fig:2ndFreezerModelSimulation}
\end{figure}


\subsection{mFRR}\label{sec:mFRR}

mFRR is a slow-responding reserve which is activated after primary and secondary reserves in order to restore the frequency in the power grid to 50 Hz. The market for mFRR is usually operated by each country's respective TSO.\footnote{In Denmark, the TSO is Energinet.}

Figure \ref{fig:timeline_mfrr} shows the timeline of the mFRR market in Denmark.\footnote{There is only a market for up-regulation.} First, BRPs can bid reserve capacities in each hour, $p_{h}^{r,\uparrow}$ $\forall{h} \in \{1, \ldots 24 \}$, in the market for the next day, $D$. If accepted, they receive the reservation price, $\lambda_{h}^{r,\uparrow}$. This happens \textit{before} the day-ahead market clearing for which the BRPs buy energy for their expected demand, $P_{h}^{\text{Base}}$, at the spot price, $\lambda_{h}^{s}$. After that, a regulating power bid, $\lambda_{h}^{\text{bid}}$, must be submitted for each hour in $D$ where $p_{h}^{r,\uparrow} > 0$. In real-time, the reserves are activated if the following conditions hold, depending on the balancing price, $\lambda_{h}^{b}$:

\begin{itemize}
    \item $p_{h}^{r,\uparrow} > 0$
    \item $\lambda_{h}^{\text{bid}} <  \lambda_{h}^{b}$
    \item $\lambda_{h}^{b} > \lambda_{h}^{s}$
\end{itemize}

\begin{figure}[!t]
    \centering
    \includegraphics[width=\columnwidth]{../figures/timeline_mfrr.png}
    \caption{Timeline of the Danish mFRR market.}
    \label{fig:timeline_mfrr}
\end{figure}


If the conditions are met, the BRP receives the balancing price times their actual up-regulation, $p_{h}^{b,\uparrow}$. The BRP also incurs an additional cost due to any subsequent rebound. Furthermore, the BRP incurs a penalty, $s_{h} = \text{max}\{0, p_{h}^{r,\uparrow} - p_{h}^{b,\uparrow}$\}, if they don't deliver their promised reserve.

A suitable objective function for a BRP delivering mFRR up-regulation for one day is therefore:

\begin{align}\label{eq:mFRRObjective}
     & \text{Objective mFRR} = - \underbrace{\sum_{h=1}^{24} \lambda^{s}_{h}P^{\text{Base}}_{h}}_{\textrm{Energy cost}} + \underbrace{\sum_{h=1}^{24}\lambda_{h}^{r} p^{r, \uparrow}_{h}}_{\textrm{Reservation payment}}  \notag \\ & \quad \quad + \underbrace{\sum_{h=1}^{24}  \lambda_{h}^{b} p^{b,\uparrow}_{h}}_{\textrm{Activation payment}} - \underbrace{\sum_{h=1}^{24}  \lambda_{h}^{b} p^{b,\downarrow}_{h}}_{\textrm{Rebound cost}} - \underbrace{ \sum_{h=1}^{24}  \lambda^{p}s_{h}}_{\textrm{Penalty cost}}
\end{align}


\subsection{Load shifting}

Another option for utilizing flexibility is to shift the load to a different time according to the spot prices which are known already 12-36 hours in advance. Then it is simply a matter of consuming in low-price hours and not in high-price hours.

For a TCL, there are additional constraints to how the energy can be shifted and for the rebound. First, there can be temperature constraints which will result in less energy being shifted. Second, the rebound must happen immediately before or after reducing power consumption (otherwise, the temperature deviation becomes too big).

The savings from load shifting are directly proportional to the volume and price difference between the shifted load and baseline load as given by:

\begin{equation}\label{eq:load_shifting_savings}
    \text{Load shifting savings} = \sum_{h=1}^{24} \lambda^{s}_{h} p^{\text{Base}}_{h} - \lambda^{s}_{h} p^{\prime}_{h}
\end{equation}

where $p^{\prime}_{h}$ is the flexible power profile, $p^{\text{Base}}_{h}$ is the baseline power, and $\lambda^{s}_{h}$ is the spot price.

However, since the load shifting action only occurs \textit{after} the day-ahead market clearing (cf. Figure \ref{fig:timeline_mfrr}), the BRP has already bought $\lambda^{s}_{h} p^{\text{Base}}_{h}$ and any deviation results in an imbalance for the BRP. In this work, we look at the case where the flexible demand acts selfishly and excludes the BRP from its load shifting action. Therefore, the objective function for the flexible demand is simply:

\begin{equation}\label{eq:LoadShiftingObjective}
    \text{Objective load shifting} = \sum_{h=1}^{24} \lambda_{h}^{s} p_{h}^{\prime}
\end{equation}
