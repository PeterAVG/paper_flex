\vspace{-2mm}
\section{Conclusion}\label{sec:conclusion}
%
We investigated how a supermarket freezer can provide flexibility for mFRR and load shifting in Denmark, and explored which one provides a greater monetary incentive for a flexible load. To this end, we used actual data  from a Danish supermarket. This was done by developing a second-order grey-box model of the temperature dynamics in the freezer with the food temperature as a latent state. In the state-space form, the model was directly incorporated as constraints into a two-stage stochastic MILP problem, whose objective is to maximize the monetary value from the freezer's flexibility. Two scenario generation strategies were implemented: one with a five-day lookback strategy on the spot and balancing market prices in DK2, and the other one  based on price data for those markets in 2021. For mFRR, we used a linear policy, and then linearized the conditions for activation via the McCormick relaxation method. For computational ease, we used an ADMM-based scenario decomposition technique.  An out-of-sample evaluation was done on unseen 2022 price data. We observed that load shifting is more profitable, but has a greater impact on the air and food temperatures in the freezer as opposed to mFRR that depends on the system state and bid price for activation.

We made a set of simplifications and assumptions, whose impacts need to be explored for the future work. The revenue share of BRP and aggregator was not considered. This may change our finding that load shifting is more financially appealing to a flexible load. As mentioned earlier, a single freezer or supermarket must be part of a larger portfolio through an aggregator in order to participate in the mFRR market. Such an aggregated portfolio has some issues that are neglected here, such as the baseline estimation for verification of the demand response, allocation of profits within the portfolio, and an accurate capacity estimation of the whole portfolio that bids in the mFRR market. Furthermore, the European mFRR markets will change from a 60-minute resolution to a 15-minute market in the next few years \cite{MARI}. This makes it more feasible for TCLs to participate in the mFRR market, given their sensitivity to large temperature deviations, and the fact that TCLs can get a passive income when there is no up-regulation need in the power grid.