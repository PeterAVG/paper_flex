\section{Conclusion}\label{sec:conclusion}

In this paper, it has been investigated how a supermarket freezer can provide flexibility for mFRR and load shifting in Denmark to see which one provide the greater incentive for a flexible load. To this end, actual data was used from a Danish supermarket. This was done by creating a second-order grey-box model of the temperature dynamics in the freezer with the food temperature as a latent state. In state-space form, the model was directly embedded as constraints in a two-stage stochastic MILP which maximizes value from the freezer's flexibility. Two solution strategies was implemented: 1) one with a five day lookback on spot prices and 2) one where the policy was learned on 2021 using up to 250 scenarios and solved using ADMM decomposition. For mFRR, a the sequence of decisions was implemented using McCormick relaxation. The optimization model simplified to a simple MILP for load shifting and could be solved directly. Evaluation was done on unseen 2022 price data. Load shifting was more profitable, but had a greater impact on the temperatures in the freezer as opposed to mFRR that depends on the system state and bid price for activations. However, the BRP and aggregator share of the revenue was not considered. Hence, we find that load shifting is more appealing for a flexible load which is not necessarily beneficial for the TSO and the power grid.
